%% -*- mode: poly-noweb+R; ess-indent-level: 2; -*-

\chapter{Teendők \code{R} verzió váltásnál}\label{chap:9}

\section{\code{R} csomagok installálása forrásból, 
\code{type="source"}}\label{sec:9.1}

Windows alatt a csomagokat jellemzően \code{zip file}-ból telepíti az
ember. Ennek legfőbb előnye, hogy nem szükséges fordító programot
telepíteni a gépre. Másfelől, mégis célszerű lehet forrásból telepíteni, abban az esetben, ha a csomag \code{zip} változata valamilyen ok miatt nincs fenn
a \code{CRAN}-on. Ha ragaszkodunk a menü használatához és mégis
forrásból akarunk telepíteni, 
akkor célszerű a \code{pkgType} opciót átállítani \code{"both"}-ra. Ez
megtehető az alábbi módon. 
\begin{Rnw}
<<eval=FALSE>>=
options(pkgType = "both")
@
\end{Rnw}
Ha a csomag forrása az esetleg szükséges \code{Rtools}-zal együtt
megtalálható a gépünkön, 
akkor az \code{R} konzolból a következő képpen indíthatjuk a telepítést:
\begin{Rnw}
<<eval=FALSE>>=
install.packages("<csomag elérési útja>", type = "source", repos = NULL)
@
\end{Rnw}
Ha a csomagban nincs fordítandó kód (\code{C/C++}, vagy
\code{Fortran}), akkor a fenti módszer az 
\code{Rtools} telepítése nélkül is működik.

Az \code{Rtools} (\code{Windows toolset}) készlet letölthető az R
projekt honlapjáról: 
\begin{center}
  \url{http://CRAN.R-project.org/bin/windows/Rtools/}.
\end{center}
A fordító készletről az előbbi linken ill. az ``R Installation and
Administration'' című leírásban lehet további részleteket találni. Ez
utóbbi az R help-jéből elérhető \code{pdf}, vagy \code{html} formátumban.
 

\subsection{\code{com}, \code{comproxy} fordítása}
A \code{com} csomag, jelenleg csak 32-bites módban fordítható. Ezért a
telepítésekor, vagy frissítésekor a  \code{INSTALL\_opts} argumentumra
is szükség van. 
\begin{Rnw}
<<eval=FALSE>>=
install.packages("<com tar.gz elérési útja>", type = "source", repos = NULL, INSTALL_opts = "--no-multiarch")
@
\end{Rnw}
A \code{com} csomagnál az utolsó argumentum nélkül is végigmegy a
telepítés, mert csomag forrásában nem üres a \code{configure.win} file
van. Nekem úgy tűnik, hogy ez egy nem dokumentált viselkedés, vagyis a
későbbi verziókban változhat. 

Ha az utolsó \code{INSTALL\_opts = "--no-multiarch"} argumentumot is
megadjuk, akkor az \code{R} csak a 
fő architektúrára, vagyis 32-bites \code{R} esetén csak 32-bites módra
próbálja fordítani a csomagot. 
Mivel az \code{Rxls} csomag a \code{com} csomagra épül, annak telepítése
során is szükség van erre %INSTALL_opts="--no-multiarch" 
az argumentumra. Ez a csomag nem tartalmaz fordított kódot, így egy
nem üres \code{configure.win} file elhelyezése nem segít. 

\subsection{\code{com} és \code{comproxy} 64 bites változata}

A \code{com} csomag \code{1.0-68} ill. a \code{comproxy}
\code{1.0-16}-os verziójától kezdve mindkét csomag használható 64
bites \code{R}-rel is és a telepítés során az
\code{INSTALL\_opts="--no-multiarch"} opcióra nincs szükség.
Azaz a forrásból történő telepítéshez a következő \code{R} parancs használható:
\begin{Rnw}
<<eval=FALSE>>=
install.packages("<com tar.gz elérési útja>", type = "source", repos = NULL)
@
\end{Rnw}

A módosítás során, a néhány helyen még előforduló, \code{assert(\dots)}
hívásokat lecseréltem. Az \code{assert(\dots)} hívások tipikusan olyan
ellenőrzések voltak, amik normál működés mellett nem okozhattak gondot,
viszont ha mégis, akkor a futó \code{R} példány összeomlását eredményezték.

További változás  \code{com\_1.0-68}-ban, a korábbi változathoz
képest, hogy a \code{rcom\_srv.tlb} és \code{StatConnLib.tlb} típus
könyvtárak az \code{R} telepítés 
\begin{align*}
  &\code{library/com/lib/i386}
  \intertext{mappájából a }
  &\color{outcolor}\code{library/com/binary}  
\end{align*}
mappába kerültek át. Ezt a változtatást
szintén a 64 bites változat tette szükségessé. Korábban a típus
könyvtár betöltése a \code{com.dll} mappájából történt, ami viszont attól függ, hogy 32 vagy 64 bites
\code{R}-et használunk-e. A \code{registry}-ben ezzel szemben egy
helyet definiálhatunk a típus könyvtárak helyének ugyanis a bitek számától
függetlenül ugyanazt az {interface}-t implementálja mind a 32, mind a
64 bites \code{DLL}. 

A típus könyvtár helyének változása miatt a
frissítés után a \code{comRegisterRegistry()} \code{R} függvényt le kell
futtatni egy rendszergazdai jogosultságokkal inditott \code{R}
példányban, lásd \aref{fig:1.1} ábrán. A telepítő ezt a lépést
automatikusan elvégzi, így ennek használatakor erre a lépésre nincs szükség.  

\section{Egyebek}\label{sec:9.2}

Van olyan korábban fejlesztett alkalmazás, ami ACCESS adatbázisból
adatokat másol adotokat EXCEL munkalapra. Ehhez a \code{DAO3.6}
library-t használtam. Ez nem érhető el a Windows újabb 
változataiban. Az interneten keresgélve azt találtam, hogy a jelenlegi
megfelelője: 
\begin{verbatim}
Microsoft Office 14.0 Access database engine Object Library
\end{verbatim}
Ha ez a hiba előfordul, akkor a szóbanforgó munkafüzet (pl. az \code{arviz}
csomag \code{Karkalkulator}  
munkafüzetében) a hivatkozást (\code{VB editor->References} menüpontját
használva) le kell cserélni. 

%%% Local Variables: 
%%% mode: latex
%%% TeX-master: "Rxls"
%%% End: 
