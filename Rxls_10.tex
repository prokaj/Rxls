%% -*- mode: poly-noweb+R; ess-indent-level: 2; -*-
\chapter{Újdonságok az \code{Rxls} csomagban az 1.0-119
verzióhoz képest}\label{chap:10}


Amikor ezt a leírást készítettem, tapasztalnom kellett, hogy a korábbi
változat nem működik a 64 bites EXCEL 2010 programmal. Ezt pár apró
változtatás követte, majd újra átgondoltam néhány megoldást, amit
újabb változtatások követtek. Ezért úgy döntöttem, hogy a verzió
számot 2.0-0-ra növelem, és mindent átírok, amiről úgy gondolom
jobban, egyszerűbben is megoldható.

\subsection{\code{Rxls} könyvtár}

A korábban készített munkafüzeteknél az egyik gond a hordozhatóság
volt. Minden ilyen (számoló) munkafüzet hivatkozást tartalmazott az
\code{R.xls} filera. Ez tartalmazza a szükséges \code{Visual Basic} kód nagy
részét. A gondot az okozza, hogy amikor egy másik gépre másoljuk a 
számoló munkafüzetet, vagy verzió váltás miatt a korábbi, \code{R.xls}
könyvtár megszűnik, akkor hivatkozás elromlik, és azt kézzel kellett javítani.
Erre két megoldás látszik. Az egyik, hogy az \code{R.xls} állományt az \code{R}
könyvtár struktúráján kívül tároljuk, a felhasználó meghatározott
könyvtárában, pl. \code{\{APPDATA\}/Rxls/R.xls}. A 
hivatkozó munkafüzet tudja a relatív helyet az \code{\{APPDATA\}} könyvtárhoz
képest, amit pedig ki tud olvasni egy környezeti változóból. A számoló
munkafüzet megnyitáskor ellenőrzi, hogy a hivatkozott \code{R.xls}
állomány meg van-e nyitva, ha nincs megkísérli a standard helyről 
betölteni. Ennek eredményeként a számoló munkafüzet hordozhatóvá
válik; ha átvisszük 
egyik gépről a másikra, vagy verzió váltás történik, továbbra is
használható marad feltéve, 
hogy az \code{Rxls} csomag megfelelően van telepítve.

Ez tehát két ponton jelent változást: egyrészt minden számoló
munkafüzetnek tartalmaznia kell a megfelelő  \code{Workbook\_open}
eljárást, másrészt az \code{R} oldalon \code{Rxls} csomag betöltésekor 
az \code{R.xls} állományt át kell másolni a szokásos helyére, ha az
nincs ott. Ha forrásból telepítjük 
a csomagot, akkor az átmásolás a telepítés részeként is történhet.

Hasonló gond jelentkezhet az \code{RCOM 1.0 Type Library}-val
kapcsolatban is. Itt a megoldás az, hogy telepítés után
rendszergazdaként kell elindítani az \code{R} programot és ki kell adni a 
következő utasítást:
\begin{Rnw}
<<eval=FALSE>>=
comRegisterRegistry()
@
\end{Rnw}
A másik lehetőség a hordozhatóság elérésére az lehet, hogy a
véglegesítés során a szükséges modulokat bemásoljuk a számoló munkafüzetbe.

\section{\code{com} csomag}\label{sec:10.1}
Ennek az útmutatónak az írása közben vettem észre, hogy tulajdonságok
lekérdezésénél csak pozicionális argumentum átadásra van lehetőség,
név szerintire nem. Ez a 1.0-49 verziótól 
kezdve annyiban módosul, hogy tulajdonságot kétféleképpen is le lehet
kérdezni. A régi szintaxis
\begin{verbatim}
<obj>[["<tulajdonság név>", <argumentumok>]],
\end{verbatim}
továbbra is használható. Ekkor azonban hiába vannak az argumentumok névvel ellátva, azokat pozíciójuk szerint adjuk át. Korábban ez volt a viselkedés.

Emellett a metódusokra megszokott szintaxis is használható
\begin{verbatim}
<obj>$<tulajdonság név>(<arg1>=<value1>,...).
\end{verbatim}
Ez utóbbi a \code{<obj>.<tulajdonság név> <arg1>:=<value1>, ...}
Visual Basic szintaxisnak felel meg. Ugyanúgy, ahogy Visual Basic-ben
itt sem szükséges neveket megadni, viszont a zárójel akkor is kell, ha
nincs argumentum. 

\subsection{\code{comCheckRegistry}}
\code{TRUE} értéket ad vissza, ha a \code{Registry} tartalmazza a \code{RCOM}
szerverre mutató bejegyzést ellenkező esetben az érték
\code{FALSE}. Arra szolgál, hogy a telepítés állapotát ellenőrizni
lehessen. 
\begin{Rnw}
<<>>=
comCheckRegistry()
@
\end{Rnw}
%> comCheckRegistry()
%[1] TRUE
Egészen pontosan a következők ellenőrzése történik meg: regisztrálva
van-e az \code{R COM} 
szervere, regisztrálva van-e a \code{COM} szerver \code{type
  library}-ja, végül az \code{R} 
legfrissebb változata 
alól van-e regisztrálva. Ha valamelyik pont esetében a regisztry
bejegyzés nem létezik, vagy 
nem az, aminek lennie kellene a felhasználó figyelmeztetést kap a
teendőkkel együtt. 

\subsection{\code{RODBC}}

A \code{CRAN}-ról letölthető \code{RODBC} csomag
\code{odbcConnect} függvénye másképp viselkedik 
\code{RGui}-ból használva, mint egyébként. Ha \code{Rgui}-ból indítjuk
paraméterek nélkül, akkor egy felugró ablakból kiválaszthatjuk a
használandó adatbázist, megadhatjuk a felhasználó nevet és a jelszót,
ha ez szükséges, stb. Egyéb formáját használva az \code{R}-nek
ugyanez az eljárás hibát eredményezne.

A munka során átadtam egy patchelt változatot az \code{RODBC}
csomagból, ami ezt a különös viselkedést orvosolja. A javasolt
változtatást a csomag karbantartójának is jeleztem, de eddig nem
vezette át, így valószínűleg nem is fogja. 

Ahhoz, hogy a csomag legfrissebb változátával is használhassuk a
módosítást a következőt kell tenni. 
\begin{itemize}
\item Le kell tölteni a csomag forrás változatát. (\code{RODBC...tar.gz})
\item Kibontás után az \code{src/RODBC.c} fileban a következő
  változásokat kell átvezetni:

A file elején a \code{\_\_declspec(dllimport) window RConsole;} sort
le kell cserélni és néhány sort kikommentezni.
 Csere után ez a rész a következőképpen néz ki:
\begin{VBAframe}
#ifdef WIN32
# include <windows.h>
# undef ERROR
// comment out the next few lines in the original source and include 
// the R_interactive line below.  
/* enough of the internals of graphapp objects to allow us to find the
   handle of the RGui main window */
//typedef struct objinfo \{
//	int	kind, refcount;
//	HANDLE	handle;
//\} *window;
//__declspec(dllimport) window RConsole;
{\color{DarkRed}__declspec(dllimport) extern int R_Interactive;}
#else
# include <unistd.h>
#endif
\end{VBAframe}

Az \code{RODBCDriverConnect} függvény elejét módosítani kell a kommentekkel
megjelölt helyeken:
\begin{VBAframe}
SEXP RODBCDriverConnect(SEXP connection, SEXP id, 
                        SEXP useNRows, SEXP ReadOnly)\{
    SEXP ans;
    SQLSMALLINT tmp1;
    SQLRETURN retval;
    SQLCHAR buf1[buf1_len];
    pRODBCHandle thisHandle;

    PROTECT(ans = allocVector(INTSXP, 1));
    INTEGER(ans)[0] = -1;
    if(!isString(connection)) \{
       warning(_("[RODBC] ERROR:invalid connection argument"));
       UNPROTECT(1);
       return ans;
    \}
    thisHandle = Calloc(1, RODBCHandle);
    ++nChannels;

    odbcInit();
    retval = SQLAllocHandle(SQL_HANDLE_DBC, hEnv, &thisHandle->hDbc);
    if(retval == SQL_SUCCESS || retval == SQL_SUCCESS_WITH_INFO) \{
      if(asLogical(ReadOnly))
        SQLSetConnectAttr(thisHandle->hDbc, SQL_ATTR_ACCESS_MODE, 
          (SQLPOINTER) SQL_MODE_READ_ONLY, 0);{\colorlet{textcolor}{DarkRed} 
         //insert the next three lines 
          //it gets the desktop window handle.
#ifdef WIN32
         HWND desktopHandle = GetDesktopWindow();
#endif}
         retval =
           SQLDriverConnect(thisHandle->hDbc, {\colorlet{textcolor}{DarkRed}
#ifdef WIN32
             // insert the next line and comment the original
             R_Interactive? desktopHandle : NULL, 
             //RConsole ? RConsole->handle : NULL,
#else
             NULL,
#endif
}            /* This loses the const, but although the
                declaration is not (const SQLCHAR *),
                it should be. */
             (SQLCHAR *) translateChar(STRING_ELT(connection, 0)),
             SQL_NTS,
             (SQLCHAR *) buf1,
             (SQLSMALLINT) buf1_len,
             &tmp1,{\colorlet{textcolor}{DarkRed}
#ifdef WIN32
               // insert the next line and comment the original
               R_Interactive ? SQL_DRIVER_COMPLETE : SQL_DRIVER_NOPROMPT
#else
               SQL_DRIVER_NOPROMPT
#endif}
             );
\end{VBAframe}
A rutin többi részét nem kell változtatni.
\item Az így módosított file-t a kibontott csomag többi részével
  együtt vissza kell csomagolni,  
  pl. a \code{RODBC.tar.gz} név alatt.
\item Forrásból kell telepíteni a csomagot a következő paranccsal.
\begin{verbatim}
install.packages("RODBC.tar.gz", repos=NULL, type="source")
\end{verbatim}

\end{itemize}

\section{Bizonytalan kapcsolatfelvétel \code{Rstúdió}val, 
2014-03-11}
\label{sec:10.2}

\code{EXCEL}-ből kezdeményezve az összekötést egy futó \code{Rstúdió}
példánnyal előfordulhat, hogy a kapcsolat létrehozása sikertelen. Ez
nálam nem jelentkezett, de az \code{Rstudio} 0.98.501 változatát
telepítve én is megtapasztaltam. A hiba okát nem sikerült teljesen
felderíteni, de annyi kiderült, hogy a \code{R.xls} munkafüzet
\code{RIC} moduljában a \code{cmdToR} eljárásában kell keresni.
A következő változat nálam gond nélkül működik. A változtatás a
\code{.useSendKey} ágon belül \code{SendKeys "\{ESC\}", True} sorban
történt. Korábban itt \code{"\{ENTER\}"} szerepelt. 
\begin{VBAframe}
Private Sub cmdToR(cmd)
  On Error GoTo Errorhandler
  With curRproc
    If .useSendkey Then
      Dim wstate
      wstate = WINAPI.WindowState(.hwnd)
      WINAPI.ShowWindow .hwnd, WINAPI.SW_NORMAL
      AppActivate curRproc.pid
      If VBA.Len(cmd) > 1 And .useClipboard Then
        Dim x As New DataObject
        SendKeys .sendkeyPrefix, True
        If .delay > 0 Then
          DoEvents
        End If
        x.Clear: x.SetText cmd: x.PutInClipboard {\colorlet{textcolor}{DarkRed}
        SendKeys "\{ESC\}", True }
        SendKeys .ctrlv, True
        SendKeys "\{ENTER\}"
        DoEvents
        WINAPI.ShowWindow .hwnd, wstate
      Else
        SendKeys .sendkeyPrefix
        SendKeys "\{ENTER\}"
        SendKeys quote(cmd)
        SendKeys "\{ENTER\}"
      End If
    Else
      WINAPI.SendText vbCrLf & cmd & vbCrLf, .hwnd
    End If
  End With
  Exit Sub
Errorhandler:
  Dim errnum, errtxt
  errnum = err.Number: errtxt = err.Description
  err.Clear
  Select Case errnum
    Case 91 ' obj does not exist
      If allRproc Is Nothing Then RprocInit: Resume
      If allRproc.Count = 0 Then Rstart: Resume
    Case 5 '??? mi a kodja annak amikor ay appactivate okozza a hibat
      With curRproc
        If (WINAPI.IsWindow(.hwnd) > 0) And _
           (WINAPI.IsWindowVisible(.hwnd) <= 0) Then
          WINAPI.ShowWindow .hwnd, WINAPI.SW_SHOWMINIMIZED
          Resume
        End If
      End With
  End Select
  MsgBox "Nem lehet írni az R konzolba!" & vbCrLf & _
          errtxt & "[errnum=" & errnum & "]", vbOKOnly
  End
End Sub
\end{VBAframe}


\section{VBA módosítás, 2014-03-11}

Az \code{R.xls} munkafüzet \code{RIC} modulja kapott egy szubrutint
\code{showRprocs} névvel.  Ez megjeleníti a futó \code{R}
példányokat. Az \code{Rxls development} fül alá beraktam
egy gombot, amivel ez a funkció kényelmesen elérhető.
Ugyanebben a modulban az \code{RConnectedToThis} és 
 \code{RnotBusy} függvénynek adtam egy
opcionális \code{Rproc} típusú argumentumot, ha nem adjuk meg, akkor a
\code{curRproc} 
példányt használják a rutinok, különben a megadottat. A \code{selectR} dialóg
gombjait pedig felcseréltem, felülre raktam a \code{Cancel} gombot.

\section{\code{VBA} módosítás, 2014-05-13}

A \code{Calc.xlt} sablont atneveztem \code{Calc.xltm}-re így a számoló
munkafüzetek ezentúl nem kompatibilis módban jönnek létre. Az
\code{Rdev.xlam} bővítmény \code{insertSkeleton} moduljában a
\code{newSkeleton} rutinjából a mentés részt kivettem. Viszont a
\code{Calc.xltm} beillesztettem egy \code{Workbook\_beforeSave} esemény
kezelőt. Így, amikor egy újonnan létrehozott  számoló munkafüzetet
először mentünk el, annak típusa makróbarát munkafüzet lesz.   
A \code{SaveAs} dialóg nem \code{.xlsx}, hanem
a \code{.xlsm} kterjesztést kínálja fel alapértelmezésben.  

\section{\code{R 3.1} változat 
\code{LENGTH or similar \dots}
%applied to externalptr object}\\
 hibaüzenet, 2014-09-01}

A hibaüzenetet az okozta, hogy a \code{com} és \code{comproxy}
csomagok \code{C} kódja az \code{R API} \code{LENGTH} függvényét
használja
% került 
több ponton.
%alkalmazásra
 Az \code{R} 3.1-es változatában, újdonságként, ebbe a
függvénybe belekerült egy ellenőrzés, ami 
\code{error} üzenetet ad, ha a függvényt nem vektor típusú
\code{SEXP}-re alkalmazták. A \code{LENGTH} mellett az \code{R
  API}-ban van egy másik függvény is az \code{Rf\_length}, amit a
bővítmény csomagokban javasolnak alkalmazni. 


A  következő \code{R} szkripttel lehet megnézni a két függvény
viselkedését. A kódrészlet végén az új, 3.1-es \code{R} alatt látszik
a futási eredmény. 

\begin{Rnw}
<<prompt=FALSE,message=FALSE,warning=FALSE,tidy=FALSE>>=
require(inline)
require(Rcpp)
funs<-c("LENGTH","Rf_length")
names(funs)<-paste("test",funs,sep="_")
body.pat<-'
  SEXP sexp = R_NilValue;
  sexp = R_MakeExternalPtr(NULL,R_NilValue,R_NilValue);
  return IntegerVector::create( _["%s"]=%s(sexp));
'

body<-lapply(funs,function(x)sprintf(body.pat,x,x))

test<-cxxfunction(sig=lapply(funs,"[",0), body=body, plugin="Rcpp")
names(test)<-funs

for(x in names(test)){
  cat(sprintf("Running 'test$%s':\n result: ",x))
  cat(eval(substitute(try(test[[x]](),silent=T),list(x=x))))
}
@ %$
\end{Rnw}

Az \code{R} korábbi változataiban az eredmény \code{LENGTH} esetén 0,
míg \code{Rf\_length} esetén 1 lett volna.
Ez azért nem okozott gondot, mert
ahol  \code{externalptr} object előfordulhatott ott valójában azt
nézte a program, hogy a hosszra valami teljesül-e, vagy
\code{externalptr}-ről van-e szó. 

A \code{C}, \code{C++} forrásokban kicseréltem a \code{LENGTH}
függvényhívásokat \code{Rf\_length}-re. Ezzel a problémát sikerült
orvosolni.
 
\section{\code{VBA} módosítás, 2014-09-01}

A kapcsolatfelvétel az \code{Rstúdió}val nem megnyugtató. Ha még semmi nem
került az \code{R} consolba, akkor akár \code{\{ENTER\}}, akár \code{\{ESC\}}
leütést küldünk, az \code{Rstúdió} leáll és új \code{R
  session}t kell indítani.
Ezért kivettem a korábban jó megoldásnak tűnő ``\code{sendkey
  \{ESC\},True}'' sort az \code{R.xls} munkafüzet \code{RIC}
moduljának \code{cmdToR} eljárásából, lásd a \ref{sec:10.2} szakaszt.

\section{\code{EXCEL} módosítás, 2015-09-01}

\begin{itemize}
\item Az \code{Rselect} űrlapban az \code{R} verzióját is
  feltüntetjük. \code{Rstudió} esetén csak akkor, ha a \code{com}
  csomag be van töltve.
\item Javítás az \code{Rproc} modulban. Ha az \code{Rstudió} ``projekt
  módban'' dolgozik, akkor is megtaláljuk az ablak azonosítóját.  
\item A  \code{R.RIC} modul \code{connect} rutinja
  módosult. Létrehozza
  az adott EXCEL példány környezetét az \code{R} oldalon. Ha menüből
  indítjuk, akkor meghívja az \code{attachXL} függvényt is. Ezzel az \code{R}
  keresési útvonalához csatolja a \code{THISXL, .wbname, .wsname}
  változókat.
\item  A  \code{R.RIC} modul \code{Rclose} rutinja
  módosult. Törli az \code{EXCEL} példányra mutató környezeteket és az
  elérési útról is eltávolítja ezeket az adatokat. Ahol lehet a \code{COM}
  interfészt használja.
\item A  \code{R.WINAPI} modul \code{sendkeystohwnd} rutinja
  robosztusabb lett.
\end{itemize}
plusz további változtatások.
%%% Local Variables: 
%%% mode: latex
%%% TeX-master: "Rxls"
%%% End: 
