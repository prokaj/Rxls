%\NeedsTeXFormat{LaTeX2e}
%\ProvidesPackage{minilisp}[2002/10/19]

%% lista kezelő makrók 
%% \addtobegin#1#2 Az #1 lista elejéhez  hozzafűzi a #2 elemet.
%% \addtoend#1#2   Az #1 lista végéhez fűzi a #2 elemet.
%% \newlist#1      létrehozza az #1 nevű üres listát.
%% \getfirst#1#2   Az #1 lista első elemét leveszi, a művelet után az
%%                 #1 lista rövidebb -- és az első elemet  #2-nek adja
%%                 értékül, ha nem marad több elem a listán az
%%                 \ifendoflist igazza valik
%% \checklist#1    Ellenőrzi az #1 listát, ha üres, akkor az
%%                 \ifendoflist hamissá válik  
%% \mapcar#1#2#3   Az #1 lista minden elemére alkalmazza a #2
%%                 függvényt, és az eredménylistát a #3 név alatt
%%                 elmenti.
%%                 #2 egy két  változós macró, paraméterlistája
%%                 \testfn#1#2
%%                 alakú #1 az arg #2 az eredmény.
%% \mapc#1#2       Ugyanaz mint az előbb de nincs eredménylista, ezért
%%                 #2 paraméterlistaja is csak #1 alakú
%% \sortlist#1     Az #1 listat rendezi a #2 testfv. szerint
%%                 #2 parameterlistaja
%%                 #1#2#3
%%                 alakú, ha #1 elobb van mint #2 akkor a #2 testfv. a
%%                 #3{#1}{#2}-t ellenkezo esetben #3{#2}{#1}-t helyettesit.
%% 
%%\catcode`\@=11
%--------------------------------------------------------
\def\showlist#1{\message{\string#1=}\expandafter\printlist#1}
\def\printlist\bqlist#1\eqlist{\message{|#1|}}

\def\addtoend#1#2{%
  \expandafter\addtoend@#1{#2}{#1}}

\def\addtoend@#1\eqlist#2#3{\def#3{#1{#2}\eqlist}}

\def\addtobegin#1#2{% 
  \expandafter\addtobegin@#1\endaddtobegin@{#2}{#1}}

\def\addtobegin@\bqlist#1\endaddtobegin@#2#3{\def#3{\bqlist{#2}#1}}

\def\addtolist@#1\endaddtolist@#2{\addtoend#2{#1}}

\def\addtocar#1#2{%
  \lisp@save\aux@
  \getfirst#1\aux@
  \expandafter\addtocar@\aux@#2\endaddtocar@#1
  \lisp@restore}
\def\addtocar@#1\endaddtocar@#2{\addtobegin#2{#1}}

\def\appendto#1#2{\expandafter\appendto@#1#2#1}
\def\appendto@#1\eqlist#2#3{\expandafter\appendto@@#2\endappendto@@{#1}#3}
\def\appendto@@\bqlist#1\endappendto@@#2#3{\def#3{#2#1}}


%--------------------------------------------------------

\def\newlist#1{\def#1{\bqlist\eqlist}}

%%%
\newlist\lisp@save@list
\newcount\lisp@count
\def\lisp@savenum{0}

\def\inc#1{\lisp@count=#1\relax\advance\lisp@count by1\relax
  \edef#1{\number\lisp@count}}

\def\ginc#1{\lisp@count=#1\relax\advance\lisp@count by1\relax
  \xdef#1{\number\lisp@count}}

\def\dec#1{\lisp@count=#1\relax\advance\lisp@count by-1\relax
  \edef#1{\number\lisp@count}}

\def\add#1#2{\lisp@count=#1\relax\advance\lisp@count by #2\relax
  \edef#1{\number\lisp@count}}

\def\sub#1#2{\lisp@count=#1\relax\advance\lisp@count by -#2\relax
  \edef#1{\number\lisp@count}}

\def\mult#1#2{\lisp@count=#1\relax\multiply\lisp@count by #2\relax
  \edef#1{\number\lisp@count}}

\def\div#1#2{\lisp@count=#1\relax\divide\lisp@count by #2\relax
  \edef#1{\number\lisp@count}}

\def\lisp@save#1{%
  \inc\lisp@savenum
  % \global\let\lisp@savenum=\lisp@savenum
  \expandafter\lisp@save@\lisp@savenum\relax#1%
}

\def\lisp@save@#1\relax{%
  \expandafter\lisp@save@@\csname lisp@save@listitem#1\endcsname
}

\def\lisp@save@@#1#2{%
  \let#1=#2%
  \addtobegin\lisp@save@list{#2#1}%
}

\def\lisp@restore{%
  \mapfirst\lisp@save@list\lisp@let@
  \dec\lisp@savenum
  % \global\let\lisp@savenum=\lisp@savenum
}

\def\lisp@let@#1{\let#1}

\def\car#1#2{\car@#1\relax#2}
\def\car@#1#2\relax#3{\def#3{#1}}
\def\caar#1#2{\caar@#1\relax#2}
\def\caar@#1#2\relax#3{\caar@@#1\relax#3}
\def\caar@@#1#2\relax#3{\def#3{#1}}

%-----------------------------------------------------------------------
% az #1 lista elso elemet levagja es a #2 valtozoba teszi. vege jelzest ad.
\def\getfirst#1#2{\lisp@ifemptylist#1{\def#2{}}%
  {\expandafter\getfirst@#1\endgetfirst@#1#2}}

\def\getfirst@\bqlist#1#2\endgetfirst@#3#4{\def#3{\bqlist#2}\def#4{#1}}

\def\mapfirst#1#2{%
  \lisp@ifemptylist#1\relax{%
    \expandafter\mapfirst@#1\endmapfirst@#1#2%
  }%
}

\def\mapfirst@\bqlist#1#2\endmapfirst@#3#4{%
  \def#3{\bqlist#2}%
  #4{#1}%
}
%------------------------------------------------------------------------
% az egyes lista minden elemere alkalmazza a #2 fuggvenyt es az eredmenylistat
% a #3 nev alatt elmenti. #2 egy ket valtozos macro parameterlistaja #1#2
% alaku #1 arg #2 az eredmeny. 

\def\maplist#1#2#3{%
  \lisp@save\aux@next \lisp@save\aux@elist 
  \lisp@save\aux@mlist \lisp@save\aux@fv
  \newlist\aux@elist\let\aux@mlist#1\let\aux@fv#2%
  \maplist@
  \let\n@e@x@t@\aux@elist
  \lisp@restore\lisp@restore\lisp@restore\lisp@restore
  \let#3\n@e@x@t@
}

\def\maplist@{%
  \lisp@ifemptylist\aux@mlist\relax{%
    \aux@fv\aux@mlist\aux@next
    \expandafter\addtolist@\aux@next\endaddtolist@\aux@elist
    \getfirst\aux@mlist\aux@next
    \maplist@
  }%
}


\def\mapl#1#2{%
  \lisp@save\aux@elist \lisp@save\aux@mlist \lisp@save\aux@fv
  \newlist\aux@elist \let\aux@mlist#1 \let\aux@fv#2%
  \mapl@
  \lisp@restore\lisp@restore\lisp@restore\lisp@restore
}

\def\mapl@{%
  \lisp@ifemptylist\aux@mlist\relax{%
    \aux@fv\aux@mlist
    \getfirst\aux@mlist\aux@next
    \maplist@
  }%
}


\def\mapcar#1#2#3{%
  \lisp@save\aux@next
  \lisp@save\aux@mclist
  \newlist\aux@mclist
  \def\mapcar@@\bqlist{\mapcar@{#2}}%
  \expandafter\mapcar@@#1\relax
  \let\n@e@x@t@\aux@mclist
  \lisp@restore
  \lisp@restore
  \let#3\n@e@x@t@
}

\def\mapcar@#1#2{\def\lisp@dummy{#2}%
  \lisp@ifdummyeqlist\relax{%
    #1{#2}\aux@next
    \expandafter\addtolist@\aux@next\endaddtolist@\aux@mclist
    \mapcar@{#1}%
  }%
}

\def\mapc#1#2{%
  \def\mapc@@\bqlist{\mapc@{#2}}%
  \expandafter\mapc@@#1\relax
}

\def\mapc@#1#2{%
  \def\lisp@dummy{#2}%
  \lisp@ifdummyeqlist\relax{#1{#2}\mapc@{#1}}%
}

\def\incaux@n#1{\inc\aux@n}

\def\length#1#2{%
  \lisp@save\aux@n
  \def\aux@n{0}%
  \mapc{#1}\incaux@n
  \let\n@e@x@t@\aux@n
  \lisp@restore
  \let#2\n@e@x@t@
}

\def\wl@#1{\ifx#1\relax\let\next\relax\else\inc\aux@n\let\next=\wl@\fi\next}
\def\wordlength#1#2{%
  \lisp@save\aux@n
  \def\aux@n{0}%
  \expandafter\wl@#1\relax
  %\mapc{#1}\incaux@n
  \let\n@e@x@t@\aux@n
  \lisp@restore
  \let#2\n@e@x@t@
}


\def\lisp@iflength#1#2{%
  \lisp@save\aux@l
  \wordlength#1\aux@l
  \ifnum\aux@l#2\relax
  \expandafter\@firstoftwo\else\expandafter\@secondoftwo\fi
}

\def\find@elt\bqlist{\find@elt@}
\def\find@elt@end#1\eqlist{}

\def\find@elt@#1{%
  \def\lisp@dummy{#1}%
  \lisp@ifdummyeqlist{%
    \let\lisp@lastres\@secondoftwo
  }{%
    \lisp@ifdummyelt@{%
      \let\lisp@lastres\@firstoftwo\find@elt@end
    }{%
      \find@elt@
    }%
  }%
}

\def\lisp@ifmember#1#2{%
  \lisp@save\elt@\lisp@save\lisp@lastres\lisp@save#2
  \def\elt@{#1}%
  \expandafter\find@elt#2\relax
  \let\n@e@x@t@\lisp@lastres
  \lisp@restore\lisp@restore\lisp@restore
  \n@e@x@t@
}

\def\reverse@#1{\addtobegin\reverse@list{#1}}
\def\reverse#1{%
  \lisp@save\reverse@list
  \newlist\reverse@list
  \mapc#1\reverse@
  \let\n@e@x@t@\reverse@list
  \lisp@restore
  \let#1\n@e@x@t@
}

\def\bqlist{\bqlist}
\def\eqlist{\eqlist}

%az #1 listat rendezi a #2 testfv. szerint #2 parameterlistaja
%#1#2#3 alaku ha #1 elobb van mint #2 akkor #3{#1}{#2}-t ellenkezo esetben
% #3{#2}{#1} az eredmeny

\def\deflisp@ifdummy#1{%
  \expandafter\deflisp@ifdummy@\csname lisp@ifdummy#1\endcsname{#1}}

\def\deflisp@ifdummy@#1#2{%
  \expandafter\deflisp@ifdummy@@\csname #2\endcsname#1}

\def\deflisp@ifdummy@@#1#2{\def#2{%%\def\lisp@dummy{##1}%
    \ifx\lisp@dummy#1\expandafter\@firstoftwo
    \else\expandafter\@secondoftwo\fi}}

\deflisp@ifdummy{eqlist}
\deflisp@ifdummy{elt@}
\deflisp@ifdummy{empty}

\def\lisp@ifempty#1{\let\lisp@dummy#1\lisp@ifdummyempty}
\def\lisp@ifemptylist#1{\expandafter\list@ifempty@#1\endlisp@ifempty}
\def\list@ifempty@\bqlist#1\endlisp@ifempty{\def\lisp@dummy{#1}%
  \lisp@ifdummyeqlist}

\def\bubblesortlist#1#2{%
  \lisp@ifemptylist#1{}{%
    \lisp@save\aux@elt\lisp@save\aux@lst\lisp@save\aux@ered\lisp@save\aux@fn
    \let\aux@lst#1\let\aux@fn#2\newlist\aux@ered
    \expandafter\bubble@sortlist@\aux@lst
    \let\n@e@x@t@\aux@ered
    \lisp@restore\lisp@restore\lisp@restore\lisp@restore
    \let#1\n@e@x@t@
  }%
}
 
\def\bubble@sortlist@\bqlist{%
  \let\@ifneedmore\@secondoftwo
  \bubble@sortlist@@
}

%%\def\@ifeq#1#2{%
%\ifx#1#2\expandafter\@firstoftwo\else\expandafter\@secondoftwo\fi
%}

\def\@ifnum#1{%
  \ifnum#1\relax\expandafter\@firstoftwo\else\expandafter\@secondoftwo\fi
}

\def\bubble@sortlist@@#1#2{%
  \def\lisp@dummy{#2}%
  \lisp@ifdummyeqlist{%
    \addtoend\aux@ered{#1}%
    \@ifneedmore{%
      \let\aux@lst\aux@ered\newlist\aux@ered %%%\tracingall
      \expandafter\bubble@sortlist@\aux@lst
    }{}%
  }{%vege a listanak
    \aux@fn{#1}\first@elt\aux@fn{#2}\second@elt
    \@ifnum{\first@elt>\second@elt}{%
      \let\@ifneedmore\@firstoftwo
      \addtoend\aux@ered{#2}%
      \bubble@sortlist@@{#1}%
    }{%
      \@ifnum{\first@elt=\second@elt}{%
        \bubble@sortlist@eq{#1}{#2}%
      }{%
        \addtoend\aux@ered{#1}%
      }%
      \bubble@sortlist@@{#2}%
    }%
  }%
}

\def\bubble@sortlist@eq#1#2{%
  \addtoend\aux@ered{#1}%
  \message{Nem egyértelmű sorrend!!! \first@elt 
    ismétlődik: (#1) (#2)}%
}

  

\def\lispcaropt#1{\lisp@car@opt#10\relax}
\def\lisp@car@opt#1#2\relax{%\message{|#1|}
\lisp@car@opt@#1[\relax}
\def\lisp@car@opt@#1[#2\relax#3{%\message{|#1|}
  #3=#1\relax
}

\def\lispcar#1{\lisp@car#10\relax}
\def\lisp@car#1#2\relax#3{#3=#1\relax}
\def\lispcadr#1{\lisp@cadr#100\relax}
\def\lisp@cadr#1#2#3\relax#4{#4=#2\relax}
\def\lispcaddr#1{\lisp@caddr#1000\relax}
\def\lisp@caddr#1#2#3#4\relax#5{#5=#3\relax}
\def\lispcadddr#1{\lisp@cadddr#10000\relax}
\def\lisp@cadddr#1#2#3#4#5\relax#6{#6=#4\relax}

\def\sortbycaropt#1{\bubblesortlist#1\lispcaropt}
\def\sortbycar#1{\bubblesortlist#1\lispcar}
\def\sortbycadr#1{\bubblesortlist#1\lispcadr}
\def\sortbycaddr#1{\bubblesortlist#1\lispcaddr}
\def\sortbycadddr#1{\bubblesortlist#1\lispcadddr}

\def\sortlist#1#2{%
  \lisp@save\aux@slist
  \lisp@save\reslist 
  \lisp@save\aux@fn
  \let\aux@slist#1%
  \defaux@@fn#2%
  \newlist\reslist
  \sortlist@   
  \let\n@e@x@t@\reslist
  \lisp@restore\lisp@restore\lisp@restore
  \let#1\n@e@x@t@
}

\def\defaux@@fn#1{%
  \def\aux@@fn##1\endaux@@fn##2{#1{##1}{##2}\aux@@@fn}%
}
 
\def\aux@@@fn#1#2{%
  \def\aux@min{#1}\def\aux@max{#2}%
}

\def\sortlist@{%
  \lisp@ifemptylist\aux@slist{\relax}{\sortlist@@}%
}

\def\sortlist@@{%
  \getfirst\aux@slist\aux@min
  \mapcar\aux@slist\find@min\aux@slist
  \expandafter\addtolist@\aux@min\endaddtolist@\reslist
  \sortlist@
}

\def\find@min#1#2{%
  \expandafter\aux@@fn\aux@min\endaux@@fn{#1}%
  \let#2\aux@max
  }

%%% az egyes listat rendezi a elemek n. tagja szerinti novekvo
%%%  sorrendbe. #1=n #2 a lista. 

\newcount\first@elt
\newcount\second@elt

%\def\sortbycar#1{\sortlist#1\cpcar}

\def\cpcar#1#2#3{%
  \defcar#10\endcar\first@elt
  \defcar#20\endcar\second@elt
  \ifnum \first@elt>\second@elt 
     \def\next{#3{#2}{#1}}%
  \else
    \def\next{#3{#1}{#2}}%
  \fi
  \next
}

\def\defcar#1#2\endcar#3{#3=#1\relax}


%\def\sortbycadr#1{\sortlist#1\cpcadr}
\def\cpcadr#1#2#3{%
  \defcadr#100\endcadr\first@elt
  \defcadr#200\endcadr\second@elt
  \ifnum \first@elt>\second@elt 
     \def\next{#3{#2}{#1}}%
  \else
     \def\next{#3{#1}{#2}}%
  \fi
  \next
}

\def\defcadr#1#2#3\endcadr#4{#4=#2\relax}

%\def\sortbycaddr#1{\sortlist#1\cpcaddr}
\def\cpcaddr#1#2#3{%
  \defcaddr#1000\endcaddr\first@elt
  \defcaddr#2000\endcaddr\second@elt
  \ifnum \first@elt>\second@elt 
     \def\next{#3{#2}{#1}}%
  \else
     \def\next{#3{#1}{#2}}%
  \fi
  \next
}
\def\defcaddr#1#2#3#4\endcaddr#5{#5=#3\relax}

%\def\sortbycadddr#1{\sortlist#1\cpcadddr}
\def\cpcadddr#1#2#3{%
  \defcaddr#10000\endcadddr\first@elt
  \defcadddr#20000\endcadddr\second@elt
  \ifnum \first@elt>\second@elt 
     \def\next{#3{#2}{#1}}%
  \else
     \def\next{#3{#1}{#2}}%
  \fi
  \next
}
\def\defcadddr#1#2#3#4#5\endcaddr#6{#6=#4\relax}

%%%%%%%%%%%%%%%%%%%%%%%%%%%%%%%%%%%%%%%%%%%%%%%%%%%%%%%%%%%%%%%%%%%%%%%%%%%%%%
\def\enumlist#1{%
  \lisp@save\sorszam \lisp@save\aux@e
  \def\sorszam{0}\newlist\aux@e
  \mapcar#1\enumed\aux@e%
  \let\n@e@x@t@\aux@e
  \lisp@restore\lisp@restore\let#1\n@e@x@t@
  }

\def\enumed#1#2{%
  \inc\sorszam
  \expandafter\defres@\sorszam\relax{#1}{#2}%
}

\def\defres@#1\relax#2#3{%
  \def#3{{#2{#1}}}%
}

%%%%%%%%%%%%%%%%%%%%%%%%%%%%%%%%%%%%%%%%%%%%%%%%%%%%%%%%%%%%%%%%%%%%%%%%%%%%%%
\def\exlistsplitf#1{%
  \lisp@save\aux@e\lisp@save\@split
  \mapcar#1\splitf\aux@e%
  \let\n@e@x@t@\aux@e
  \lisp@restore\lisp@restore
  \let#1\n@e@x@t@
  }


\def\Ifnotcln#1:#2\relax{%
  \def\lisp@dummy{#2}%
  \lisp@ifdummyempty
}

\def\withcln#1:#2\relax#3\relax#4{\def#4{{#1}{#2}#3}}
%\def\wocln#1\relax#2\relax#3{\def#3{{}{#1}#2}}
\def\get@split#1#2\relax{%
  \Ifnotcln#1:\relax{%
    \expandafter\withcln\defaultex:%
  }{%
    \withcln
  }#1\relax#2\relax
}

\def\splitf#1#2{%
  \get@split#1\relax#2%
}

%%%%%%%%%%%%%%%%%%%%%%%%%%%%%%%%%%%%%%%%%%%%%%%%%%%%%%%%%%%%%%%%%%%%%%%%%%%%%%
\def\makell#1{\mapcar#1\conscell#1}
\def\conscell#1#2{\def#2{{{#1}}}}

\def\oldlistn#1{\expandafter\oldlistn@#1\endoldlistn@#1}
\def\oldlistn@#1\endoldlistn@#2{\oldlist{#1}#2}

\newcommand\genoldlist[3][ ]{%
  \lisp@save\aux@list
  \newlist\aux@list
  \def\@old@list##1#1{%
    \def\lisp@dummy{##1}%
    \lisp@ifdummyeqlist\relax{%
      \lisp@ifdummyempty\relax{%
        \addtoend\aux@list{##1}%
      }%
      \@old@list
    }%
  }%
  \@old@list#2#1\eqlist#1%
  \let\n@e@x@t@\aux@list
  \lisp@restore
  \let#3\n@e@x@t@
}


\def\oldlist#1#2{%
  \lisp@save\aux@list
  \newlist\aux@list
  \old@list#1 {\eqlist} %
  \let\n@e@x@t@\aux@list
  \lisp@restore
  \let#2\n@e@x@t@
}

\def\old@list#1 {%%\message{#1.}%
  \def\lisp@dummy{#1}%
  \lisp@ifdummyeqlist\relax{%
    \lisp@ifdummyempty\relax{%
      \addtoend\aux@list{#1}%
    }%
    \old@list
  }%
}

\def\@pti@nal#1[#2]#3\relax{\addtoend\aux@list{{#1}{#2}}}

\def\listold#1{%
  \lisp@save\ol@
  \def\ol@{}%
  \mapc{#1}\list@old
  \let\n@e@x@t@\ol@
  \lisp@restore
  \let#1\n@e@x@t@
}

\def\list@old#1{%
  \ifx\ol@\empty
     \defol@{#1}%
  \else
     \expandafter\defol@@\ol@\endol@@{#1}%
  \fi
}

\def\defol@#1{\def\ol@{#1}}
\def\defol@@#1\endol@@#2{\def\ol@{#1 #2}}

\def\expandlist#1{%
  \lisp@save\expanded@list\lisp@save\item@i\lisp@save\item@ii
  \newlist\expanded@list
  \mapc#1\expand@item
  \let\n@e@x@t@\expanded@list
  \lisp@restore\lisp@restore\lisp@restore
  \let#1\n@e@x@t@
}


\def\set@i@ii#1-#2-#3\relax{%
  \def\item@i{#1}%
  \def\item@ii{#2}%
  \def\lisp@dummy{#3}%
  \lisp@ifdummyempty{\expand@item@opt#1[]\relax}\expand@item@
}

\def\expand@item@opt#1[#2]#3\relax{\expand@item@@@#1\relax{#2}}

\def\expand@item#1{%\message{.#1}
  \set@i@ii#1-#1-\relax\relax}

\def\expand@item@{%
  \ifnum \item@i>\item@ii\relax 
     \let\next\relax
  \else
     \let\next\expand@item@@
  \fi
  \next
}

\def\expand@item@@{%
  \expandafter\expand@item@@@\item@i\relax{}%
  \inc\item@i
  \expand@item@
}

\def\expand@item@@@#1\relax#2{%
 \addtoend\expanded@list{{#1}{#2}}%
}

\def\lexpandoldlist#1#2{%
  \oldlist{#1}#2\expandlist#2%
}%\makell#2}

\def\union@#1{\union@@#1\lisp@endarg}
\def\union@@\bqlist#1\lisp@endarg{\expandafter\union@@@\aux@{#1}}
\def\union@@@#1\eqlist#2{\def\aux@{#1#2}}
\def\union#1{%
  \lisp@save\aux@\newlist\aux@%
  \mapc#1\union@
  \let\n@e@x@t@\aux@
  \lisp@restore
  \let#1\n@e@x@t@
} 

\newlist\providedfiles
\def\provide#1{%
  \addtobegin\providedfiles{#1}%
}
\def\require#1{%
  \lisp@ifmember{#1}\providedfiles
  \relax
  {%
    \input #1\relax
  }%
}

%%\catcode`\@=12
\provide{minilisp.sty}
\endinput

%%% Local Variables: 
%%% mode: latex
%%% TeX-master: t
%%% End: 
