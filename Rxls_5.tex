%% -*- mode: poly-noweb+R; ess-indent-level: 2; -*-


\chapter{Az \code{R.xls} munkafüzet részletesebben}
\label{chap:5}

A legfontosabb részek az \code{Interface}, \code{selectfuns} és a
\code{RIC} modulokban vannak. Az \code{Interface} 
modul szubrutinjaiból lehet a számolást indító ill. leállító gomb
makróit összerakni. Ezeknek számtalan verziója van, a korábbi számoló
munkafüzetekkel való kompatibilitás megőrzése 
miatt. A \code{selectfuns} modulban vannak azok a függvények, amiket a
munkalapokon szereplő 
könyvtár, ill. fileválasztó mezők gombjai használnak. A tényleges
munkát a \code{RIC} modul végzi. 
Itt vannak azok a rutinok, melyek az \code{R} példányok
nyilvántartásáért, elindításáért, leállításáért 
felelnek.

A további modulok %az előző három által igényelt 
kisegítő függvényekből állnak. A \code{Registry}
modulban, nevének megfelelően, a \code{registry} írás, olvasás rutinjai
vannak. Itt van az rutin, ami az \code{R} telepítési könyvtárat
kiolvassa a registryből, ill. ellenőrzi az \code{RCOM type library}
bejegyzés meglétét. A \code{WINAPI}  
a windows üzenetküldési mechanizmusát használó rutinok gyűjteménye. A
\code{Pick} modulban 
a könyvtár, file, ill. \code{ACCESS} tábla választó rutinok
vannak. Ezeket  a \code{selectfuns} 
modulban használjuk. A \code{worksheetfunctions} modul az \code{extAddress}
munkalapfüggvényt definiálja. Ezt 
a számoló munkafüzetek használhatják, de érdemesebb munkafüzet szintű
neveket használni helyette. Végül a \code{formadjust} modult a
\code{selectR} űrlap összeállításához használtam. 

A fenti kód modulok mellett, két \code{class} modul is van: \code{Rproc} és
\code{wndData}. Az \code{Rproc} osztály % \code{class} 
példányai egy–egy \code{R} folyamatot reprezentálnak a \code{RIC}
modulban. A \code{wndData} típust a \code{WINAPI} 
modul használja.

\section{A \code{Thisworkbook} modul}\label{sec:5.1}
Ez a modul a következő rutinokból áll.
\begin{description}
\item[\code{addRmenu}, \code{RemoveRmenu}] Létrehozza ill. törli a
  menübár \code{R} 
  menüjét. Ennek elemei az \code{R ablak megjelenítés/elrejtés}, \code{R <-> EXCEL
  összekötés}, \code{R megszakítás}. A \code{R} menü a \code{RIBBON}t
használó újabb 
  változatokban a címsor \code{Bővítmények} füle alatt található.
\item[\code{doChecks}, \code{checkRcom}, \code{checkLang}]
  Ezek ellenőrző rutinok. A \code{doChecks} a másik két
  rutint hívja meg.  Az \code{Rcom} ellenőrzése során megnézzük, hogy
  \code{RCOM type library} regisztrálva van-e.  Ha nincs, akkor felugró
  üzenetben segít abban, hogy a regisztrációt hogyan kell
  végrehajtani. A felhasználónak lehetősége van a szükséges \code{R} kódot
  vágólapra másolni és azt utána az \code{R} ablakba
  beilleszteni. Közvetlenül \code{EXCEL}-ből nem lehet a javítást elvégezni,
  mert a regisztrációt csak rendszergazdai jogosultsággal lehet
  elvégezni. 
  
  A nyelvi ellenőrzés célja, hogy az ékezetes karakterek
  a \code{COM} mechanizmuson keresztül változatlanul kerüljenek át az
  \code{R}-be 
  és fordítva. Az ellenőrzés nagyon egyszerű. Az \code{R.xls} munkafüzet
  egyetlen munkalapjának \code{\$A\$1} mezőjében fel vannak sorolva a magyar
  ABC ékezetes karakterei. Ezt egy \code{VBA} változóba másoljuk. Mivel ennek
  során már a \code{COM} mechanizmusa működik, ha nyelvi beállítások nem
  megfelelőek, akkor az \code{őű} betűk helyett \code{ou}-t kapunk. Hiba esetén
  a felugró ablak tájékoztat arról, mit kell átállítani.

  Előfordulhat, hogy a vezérlőpult felépítése
  némileg változik. Ekkor a felugró ablak szövegét, ami a modul
  elején lévő szöveges konstans célszerű átírni. A lényeg, hogy az 
  unicode szabványt nem támogató programok nyelvét kell magyarra
  állítani. Ez első sorban angol (vagy egyéb nem magyar) nyelvű
  Windows rendszereknél lehet gond. 
\item[\code{Workbook\_BeforeClose}, \code{Workbook\_AddinUninstall}]
  Bezáráskor az \code{R} menüt eltávolítjuk és az
  adott \code{EXCEL} példány által elindított \code{R} példányokat megpróbáljuk
  leállítani a \code{RIC} modul \code{CloseRcon} eljárásával. Egészen pontosan a
  következő történik. Végigmegyünk a futó \code{R} példányokon.  

  Ha valamelyik \code{R} példány ablaka rejtett, és a \code{com} csomag
  nincs betöltve, 
  akkor láthatóvá tesszük az ablakot. Így a felhasználónak lehetősége
  lesz kilépni az \code{R} alkalmazásból és bezárni az ablakot.  Ha az
  adott \code{R} példányt a mi \code{EXCEL} példányunk indította el,
  akkor megnézzük 
  számol-e még. Ha igen akkor a felhasználó dönthet, hogy
  megpróbálja-e leállítani az éppen folyó számolást.  
  
  Igen válasz
  esetén, megszakítjuk a számolást és leállítjuk a \code{R} példányt.  Nem
  válasz esetén csak a kapcsolatot reprezentáló körnezetet töröljük
  %\code{THISXL} változót
  %töröljük 
  és kilépünk az \code{EXCEL}ből. Ez a legproblémásabb
  eset. Előfordulhat, hogy az éppen futó R kódban más hivatkozás
  is van az 
  \code{EXCEL} példányunkra, vagy annak valamely részére. Ilyenkor az
  \code{EXCEL}-ből látszólag kilépünk, de az a háttérben tovább fut. 

  A  felhasználó választhatja a Mégse gombot is. Ekkor az \code{EXCEL}
  leállítása megszakad.  

  Abban az esetben, ha egy R példányt nem a mi
  \code{EXCEL} példányunk indította el,  de van a mi \code{EXCEL}
  példányunkra mutató környezet, akkor csak ezt töröljük.
\item[\code{Workbook\_Open}] Elvégzi az \code{RCOM} és a nyelvi
  beállítás ellenőrzését és 
  beállítja az \code{R} menüt.
\item[\code{Rmenuset}] Attól függően, hogy az aktuális \code{R}
  példány ablaka látható vagy rejtett (esetleg nem létezik) vált az
  \code{R} menü \code{R ablak elrejtése} ill. \code{R ablak megjelenítése} lehetőségei 
  között.
\item[\code{isXP}] Ez a függvény igaz értéket ad vissza, ha az operációs
  rendszer Windows XP. Jelenleg nem használjuk.
\end{description}
%\endinput
\section{A \code{Pick} modul}\label{sec:5.2}

\begin{description}
\item[\code{PickACCESSTable(defaultDir, file)}] A megadott
  \code{ACCESS} adatbázis 
  tábláinak listájából választhat a felhasználó. Ehhez elindítunk egy
  \code{ACCESS} példányt, betöltjük az adatbázist, majd \code{selectMDBtable}
  űrlapot kitöltjük, ebből választhat a felhasználó. 
\item[\code{PickFile(xfilter As String, defaultDir As String, Optional title)}]
  File választás.  Lehetőség szerint az OFFICE beépített file
  választó dialógját használjuk. Az \code{xfilter}  sztring
  \code{"leírás\_1,minta\_1,leírás\_2,minta\_2,...,leírás\_n,minta\_n"}
  alakú, ahol 
  a minta \code{*.<ext>} valamilyen kiterjesztéssel. Ez lehet
  pl. \code{*.xl*} is.
\item[\code{GetDirectory(initDir, Msg)}] Ez a könyvtár választó
  rutin. Az \code{initDir} 
  a kezdő könytár,  míg az \code{Msg} az felugró ablak címsora. Lehetőség
  szerint az OFFICE szokásos file választó dialógját használjuk,
  korábbi változatok esetén pedig a windows \code{shell.application
  browseforfolder} dialógját (a \code{GetDirectorySH} rutinban). Az utóbbi
  eset a mai rendszereken gyakorlatilag nem fordul elő; ha mégis,
  akkor a kezdő könyvtár argumentumot  nem használjuk.
\end{description}

\section{Az \code{Interface} modul}\label{sec:5.3}

Ebben a modulban vannak az \code{R} kód futtatására szolgáló
rutinok. Jellemzően ezek közül kerül 
ki a futtatást indító gomb makróhozzárendelése.
\begin{description}
\item[\code{RunRmain()}] Az új számoló munkafüzetek ezt használják a számolás
  indítása gomb makrójaként. Az \code{R} kód munkalapon van az \code{Rmain} nevű
  tartomány. A szubrutin az itt található  kódot küldi át az aktív \code{R}
  példánynak kiértékelésre. A kiértékelés aszinkron módon történik,
  azaz az \code{EXCEL} nem vár a számolás befejeződésére.  
\item[\code{RIgnore()}] Az aktív \code{R} példány számolásának
  megszakítása. Ehhez a \code{RIC} modul \code{stopR} rutinját
  használjuk. Ha az adott \code{R} példány számolást végez, akkor
  megjelenik egy figyelmeztető ablak.  Ha a
  felhasználó megerősíti, hogy meg akarja szakítani a számolást, akkor
  \code{ESCAPE} karaktert küldünk az \code{R} ablakba. Ennek hatására
  a számolás megszakad, gyakran némi késleltetéssel.  
\item[\code{PushCmdToR(cmd As String)}] Az \code{R} oldal \code{.FUN}
  változóját a \code{cmd}-ben megadott értékre állítja és a \code{THISXL}
  %nevű \code{R}
  változót frissíti. %  értékét 
  %szintén beállítja.
\item[\code{cmdbtnRovidFutas(Optional cmd As String = "")}] Hatására
  az \code{R} a \code{cmd} kifejezést szinkron módon 
  kiértékeli, azaz az \code{EXCEL} mindaddig nem reagál, amíg az \code{R} be nem
  fejezte a kiértékelést.
\item[\code{cmdbtnHosszuFutas}] Az előző aszinkron
  változata. Paraméterek: \code{cmd}, \code{logdev}, \code{initmsg}. 
\item[\code{RovidFutas(Optional caption As String = "*")}] A
  \code{caption} feliratú 
  gombot megkeresi az aktív munkalapon, majd a gomb alatti területben
  levő szöveget a \code{cmdbtnRovidFutas}  szubrutin segítségével átadja az
  \code{R}-nek kiértékelésre.  A korábban készült munkafüzetekben ezt
  használtam.
\item[\code{HosszuFutas(Optional caption As String = "*")}] Az előző
  aszinkron változata.
\item[\code{getButton(caption = "*", macro = "*", sheet = Nothing) As
    Shape}] Mindegyik argumentum opcionális. A \code{sheet}  
  munkalapon (ha nincs megadva, akkor az aktív lapon) megkeresi azt
  gombot (\code{shape}), melynek makróhozzárendelése és felirata illeszkedik
  az argumentumok között megadott mintára. Az összehasonlítás a \code{like}
  operátorral történik. Az újabb munkafüzetekben nem használjuk.
\item[\code{GetCmd(caption = "*", brc As range, tlc As range)}] Mindegyik
  argumentum opcionális. \code{brc} ill. \code{tlc} a \code{bottomrightcell}
  ill. \code{topleftcell} rövidítése. Ha ezek nincsenek megadva, akkor a
  \code{caption} minta alapján keresünk egy gombot az aktív munkalapon és
  ennek bal felső, ill jobb alsó sarka alatti cellára állítjuk a \code{brc},
  \code{tlc} változókat. Ezután a \code{tlc:brc} tartományban
  megkeressük az első
  nem üres cellát ennek értéke lesz a függvény visszatérési
  értéke. Ha a \code{tlc} egy egyesített tartomány eleme, akkor az egyesített
  tartomány (azaz a jobb felső sarkában lévő cellájának) értékét adja
  vissza a függvény. 
\end{description}

\section{A \code{RIC} modul}\label{sec:5.4}

A \code{RIC} (\code{R internal connector}) a legfontosabb modul az
\code{R.xls} munkafüzetben. Ez teszi lehetővé, hogy feladatokat
adjunk valamelyik futó \code{R} példánynak. A modul \code{private}, azaz 
rutinjai csak az \code{R.xls} munkafüzet számára láthatóak. A modul
legfontosabb függvénye az 
\code{Revalsync}. Ez elküldi a megadott utasítást az aktív \code{R}
példánynak. Ha nincs rendelkezésre álló \code{R} szerver, akkor hiba
generálódik, és a hiba kezelő rutin megpróbál egy alkalmas \code{R} 
példányt találni, majd újra próbálkozik az utasítás elküldésével.

A modul három globális változót használ:
\begin{description}
\item[\code{allRproc}] Ez egy \code{Rproc} collection. A változó
  inicializálásakor a 
  rendszerben lévő valamennyi  \code{R} példány adatait összegyűjtjük és itt
  tároljuk.
\item[\code{curRproc}] Az általunk használt \code{R} példány. 
\item[\code{Rmayfail}] A
  hiba kezelő rutin használja. Ha értéke \code{IGAZ}, akkor nem erőltetjük a
  feladat  végrehajtását, ha hamis, akkor hiba kezelő rutin mindent
  megtesz, hogy találjon egy  alkalmas \code{R} példányt. Az igaz értéket
  pl. akkor használjuk, ha az \code{EXCEL}ből való kilépés előtt
  szeretnénk törölni a \code{THISXL} változót az \code{R} oldalon. Ha valamilyen
  oknál fogva az \code{R} már nem fut, akkor nem kell elindítani egy új
  példányt, csak azért, hogy töröljük az éppen létrehozott \code{THISXL}
  változót.
\end{description}

A modul rutinjai:

\begin{description}
\item[\code{getAllRproc}] A Windows
  \code{winmgmts:\textbackslash\textbackslash.\textbackslash root\textbackslash cimv2} objektumát
  használva lekérdezzük valamennyi futó \code{R} példányt. Minden \code{R} példányt
  egy \code{Rproc} objektum reprezentál. Ebben számos tulajdonságot
  feljegyzünk: 
\begin{VBAframe}
Public Name As String            'A program neve: RGui, rsession, Rterm
Public RIC As InternalConnector  'Az R COMservere, ha van
Public hwnd As Long              'Az R konzol ablaka
Public outerHwnd As Long         'Az R-et futtató program ablaka
Public wndColl As Collection     'Az R példányhoz tartozó összes ablak
Public title As String           'Az ablak címsora 
Public pid As Long               'Az R processid-je
Public reportedPID As Long       'Az R Sys.getpid() hívásának értéke
Public useSendkey As Boolean     'Sendkeys függvényt kell-e használni
Public sendkeyPrefix As String   'Előtag a sendkeys függvényhez
Public delay As Long             'Késleltetés a sendkeys használatakor
Public useClipboard As Boolean   'Clipboardot kell-e használni
Public caption As String         'A cím és a PID kombinációja
\end{VBAframe}
  A fenti elemek közül a \code{useSendkey}, \code{sendkeyPrefix},
  \code{delay} és \code{useClipboard} értékeket az \code{R} process neve
  alapján döntjük el.  Végül 
  a \code{setCOMServers} eljárással az \code{RIC} értékeket gyűjtjük
  össze.  
\item[\code{setCOMServers}] Itt azt használjuk ki, hogy az \code{R}
  \code{COM} szerverét le és 
  fel tudjuk kapcsolni a \code{comRegisterServer}
  ill. \code{comUnregisterServer} 
   függvényekkel. Az eljárásunk a következő. Kérünk egy \code{rcom} szervert
  a rendszertől, ha kapunk egyet akkor lekérdezzük a \code{PID}-t és
  ez alapján az \code{rcom} szervert az \code{allRproc} collection
  megfelelő elemében 
  feljegyezzük. Ezután a szervert lekapcsoljuk és kérünk egy
  következőt, mindaddig, amíg az összes \code{rcom} szervert össze nem
  gyűjtöttük és le nem kapcsoltuk. Ezután %fordított sorrendben 
  a szervereket visszakapcsoljuk.
\item[\code{RConnectedToThis() As VbTriState}]
  Lekérdezzük az \code{R} \code{THISXL} változóját és összehasonlítjuk az \code{EXCEL}
  \code{Application} változójával.
\item[\code{RnotBusy}] Variant értékű függvény. A \code{COM}
  interfészt használja. Ha az nem áll rendelkezésre,  akkor értéke
  \code{empty}, különben \code{IGAZ} vagy \code{HAMIS}.
\item[\code{Rprocdata}] Az \code{R} szerver választó
  dialóghoz gyűjti össze az adatokat.
\item[\code{RprocInit}] Először
  összegyűjti az összes \code{R} proceszt a \code{getallRproc} függvénnyel. Ha csak
  egy \code{R} példányt találunk, akkor azt állítjuk be aktív \code{R}
  példánynak (\code{curRproc}). Ha többet is találtunk, akkor a felhasználó
  választhat. 
\item[\code{Rstart}] Ezzel a rutinnal indítunk el egy új \code{R}
  példányt. A regisztryből kérdezzük le az \code{R} telepítési útvonalát,
  és az \code{RGui} programot \code{--sdi} és \code{--vanilla}
  opciókkal indítjuk el. A 
  \code{R\_DEFAULT\_PACKAGES} környezeti változó értékét átállítjuk a
  következőre  
\begin{verbatim}
datasets,utils,grDevices,graphics,stats,methods,Rxls
\end{verbatim}
  Ez a
  szokásos esetben betöltött csomagok listája kiegészítve a \code{Rxls}
  csomaggal. Így elindulás után az \code{R} \code{COM} szervere rögtön
  rendelkezésünkre áll.  

  Indítás után 20 kísérletet teszünk arra, hogy
  az \code{allRproc} collectiont feltöltsük. Minden kísérlet között 1/10
  másodpercet várunk. Ez a beállítás az általam használt gépeken
  megbízhatóan működött. Nagyon lassú, régi konfigurációk esetén meg
  lehet növelni a várakozást, vagy az ismétlések számát.  

  Ha az \code{R}
  példányt megtaláltuk, akkor a \code{COM} szerverét próbáljuk megkapni. Ez
  csak akkor áll rendelkezésre, ha \code{com} csomag betöltése megtörtént. Ez
  némi időbe telik az \code{R}  elindulása után, ezért itt megint csak 20
  kísérletet végzünk a \code{setCOMServers} eljárással 1/10 másodperces
  késleltetésekkel. Ha sikerül a \code{COM} szerverhez csatlakozni, akkor az
  \code{R} oldalon beállítjuk a \code{.startingApp} változót a
  \code{hinstance} értékre. Ez 
  egy karakterlánc, amit az \code{EXCEL} példány \code{Hinstance} tulajdonságából
  nyerünk a \code{hinstance} függvénnyel.  
\item[\code{loadCOM}] Ezt a rutint akkor
  használjuk, ha olyan \code{R} példányhoz akarunk csatlakozni, aminek \code{COM}
  szervere nem áll rendelkezésre. Az eljárásunk hasonló a \code{Rstart}
  szubrutin második felében követetthez. Először az \code{R} példány
  ablakába írjuk a  
\begin{verbatim}
require(Rxls);comRegisterServer()
\end{verbatim}
%\code{} 
  szöveget. Ezzel
  betöltjük a szükséges csomagokat (\code{comproxy, com, Rxls}), ha azok nem
  voltak betöltve, és felkapcsoljuk az \code{COM} szervert, ha az le volt
  kapcsolva. Ezután a 20 kísérletet teszünk 1/10-ed másodperces idő
  közökkel a \code{COM} szerverhez való csatlakozásra.   Ebben a lépésben a 
  \code{setCOMServers} szubrutint használjuk.  
\item[\code{Rclose}] Az \code{R} oldalon töröljük
  az \code{EXCEL} példányra való hivatkozást. Itt az aszinkron interfészt
  használjuk. Megfontolandó, hogy kiváltható-e ez a \code{COM} szerver
  használatával.  
\item[\code{Rgetsymbol}, \code{Rsetsymbol}, \code{Reval}, \code{REvalSync}, \code{correct}]
  Ezek a függvények az aktív \code{R} példány \code{COM} szerverét használják a
  nevüknek megfelelően. A különbség az \code{Reval} és az
  \code{RevalSync}  között 
  az, hogy első a kiértékelés eredményét visszaadó függvény, míg a
  második a kiértékelés eredményét eldobó szubrutin. 

  Ezeknek a
  rutinoknak a \code{correct} hibakezelő rutin a lelke. Ha \code{COM} szerveren
  végrehajtott  művelet, valamilyen okból meghiúsul és hibát generál,
  akkor a \code{correct} rutin próbálja  helyrehozni a hibát. Ezután a
  műveletet megismételjük.  

  A következő hibákat kezeli a rutin:
  \begin{description}
  \item[{Az aktív \code{R} példány, vagy annak \code{COM} szervere
      hiányzik}] Ha az 
    \code{allRproc} \code{collection} nincs inicializálva, akkor a
    \code{RprocInit} 
    rutinnal feltöltjük azt, ha több futó \code{R} példány is van, akkor a
    felhasználó választhat aktív \code{R} példányt, stb.  
    
    Ha nincs futó \code{R}
    példány, akkor egy nulla hosszúságú \code{collection} keletkezik ebben a
    lépésben, és az ismételt próbálkozásnál ugyanerre az ágra
    kerülünk, de most már inicializált nulla hosszúságú \code{allRproc}
    \code{collection}nal. Ekkor elindítunk egy új \code{R} példányt az Rstart
    eljárással és újra próbálkozunk a művelet végrehajtásával.
    
    Ha
    van ugyan aktív \code{R} szerverünk, de annak ablaka már nem élő, akkor
    erről tájékoztatjuk a felhasználót, igény esetén
    inicializáljuk a \code{allRproc collection}t és újra próbálkozunk vagy ha
    erre nincs igény, akkor leállunk. 

    Végül, ha az aktív \code{R} ablak élő, de mégsem
    tudunk csatlakozni a \code{COM} szerveréhez,  akkor megpróbáljuk
    betölteni és felkapcsolni a \code{COM} szerverét a \code{loadCOM}
    eljárással, 
    majd újra próbálkozunk a művelet végrehajtásával.  

    \item[{Az aktív \code{R} szerver \code{COM} szervere nem elérhető}]
      Ekkor a \code{setCOMServers} rutinnal 
    lekérdezzük az elérhető \code{COM} szervereket és újra próbálkozunk a
    művelet végrehajtásával. Ez tipikusan akkor fordul elő, ha az
    \code{R}-ből kiléptünk. Ilyenkor az \code{EXCEL} oldalon olyan szerverre
    hivatkozunk, ami már nem létezik. Az ismételt próbálkozás szintén
    hibát fog generálni, de most már az első ágon fogjuk azt
    kezelni. Mivel  az aktív \code{R} példány már nem létezik, annak ablaka
    sem élő. Ez azt eredményezi, hogy a felhasználó kap egy
    kérdést, hogy akar-e új \code{R} szervert választani, stb.  
    \item[{Egyéb hiba}] A hiba szövegét egy felugró ablakban
      ismertetjük a felhasználóval és  leállunk.
  \end{description}

\item[\code{REvalAsync}, \code{cmdToR}] Az \code{REvalAsync} jelenleg
    csak egy másik 
  név a \code{cmdToR} függvényre, ami az \code{R} konzolba írja az argumentumként
  kapott szöveget. Háromféle eljárást használunk az \code{R} változattól
  függően. 
 \begin{description}
  \item[\code{RGui}] Ekkor a \code{WINAPI} modul sendtext eljárását
    használjuk, ami 
    közvetlenül a Windows üzenetküldő mechanizmusára épül.  
  \item[\code{RStudio}] Ekkor az utasítás szövegét a \code{clipboard}ra
    másoljuk, és a Windows 
    \code{Sendkeys} lehetőségét használva betűparancsok segítségével
    beillesztjük a \code{clipboard} tartalmát a konzol ablakba.  
  \item[\code{Rterm}]  Ekkor csak a \code{Sendkeys} lehetőséget
    használjuk. 
  \end{description}

\item[\code{connect}] 
  Beállítja az \code{R} oldal \code{THISXL} változójának az értékét.
\item[\code{CloseRCon}] Ezt a rutint az \code{EXCEL} leállítása során
  hívjuk meg. Ekkor 
  végignézzük a futó \code{R} példányokat.  

  Ha valamelyik \code{R} példány ablaka
  rejtett, és a \code{com} csomag nincs betöltve, akkor láthatóvá tesszük az
  ablakot. Így a felhasználónak lehetősége lesz kilépni az \code{R}
  alkalmazásból és bezárni az ablakot. 

  Ha az adott \code{R} példányt a mi
  \code{EXCEL} példányunk indította el, akkor megnézzük számol-e még. Ha igen
  akkor a felhasználó dönthet, hogy megpróbálja-e leállítani az éppen
  folyó számolást. 

  Igen válasz esetén, megszakítjuk a számolást és
  leállítjuk a \code{R} példányt.  

  Nem válasz esetén csak a kapcsolatot
  reprezentáló \code{THISXL} változót töröljük és kilépünk az \code{EXCEL}ből. 

  Ez a legproblémásabb eset. Előfordulhat, hogy az éppen futó \code{R} kódban
  más hivatkozás is van az \code{EXCEL} példányunkra, vagy annak valamely
  részére. Ilyenkor az \code{EXCEL}-ből látszólag kilépünk, de az a
  háttérben tovább fut.  

  A felhasználó választhatja a Mégse gombot
  is. Ekkor az \code{EXCEL} leállítása megszakad.  

  Abban az esetben, ha egy \code{R}
  példányt nem a mi \code{EXCEL} példányunk indította el, de erre az \code{EXCEL}
  példányra mutat a \code{THISXL} változó, akkor csak a \code{THISXL} változót
  töröljük.  

\item[\code{interruptR}] Ha van aktív \code{R} példány, akkor az \code{ESCAPE}
  karaktert írjuk az ablakába.
\item[\code{StopR}] Az \code{R} menü \code{R} megszakítás rutinja
  használja. Ha az \code{R} dolgozik, akkor rákérdezünk, hogy megszakítsuk-e
  a műveletet, és igen válasz esetén a \code{interruptR} függvénnyel ezt 
  megkíséreljük.  
\item[\code{RBringForeground}] Az aktív \code{R} példányt előtérbe
  hozza. Ha nincs aktív \code{R} példány, akkor létrehoz egyet.  

\item[\code{RPutBackground}] Ha előtérben van az aktív R példány, akkor
  elrejti az ablakát.
\item[\code{ChangeRhwndState}] Ha látható az aktív \code{R} példány akkor
  elrejti, ha nem látható akkor előtérbe hozza a
  \code{Rbringforeground} eljárással. 
\item[\code{BringAllRForeground}] Az \code{allRproc} kollekcióban
  szereplő \code{R} példányokat mind előtérbe hozza.
\item[\code{curRVisible}] Igaz értéket ad vissza, ha az aktív \code{R}
  példány ablaka nincs elrejtve. 
\item[\code{hinstance}] Egyedi azonosítót ad vissza. Azért van külön
  függvényben, mert attól függően, 
  hogy régi vagy új \code{EXCEL}ünk van más kódra van szükség.
\item[\code{screenUpdate}] Visszaállítja az \code{EXCEL}
  \code{ScreenUpdating} tulajdonságát igazra. 
\item[\code{quote}] Ha a \code{Sendkeys} hívást használjuk, akkor a
  szöveg speciális karaktereit kapcsos zárójelek közé kell tenni. Ezt
  az átkódolást végzi a rutin. 
\item[\code{dummyset}] Az \code{Rgetsymbol}, \code{Reval} függvények
  használják. Csak azért van rá szükség, mert 
  értékadáskor más kód kell objektum és nem objektum érték esetén.
\end{description}

\section{A \code{WINAPI} modul}\label{sec:5.5}

Itt vannak a Windows \code{API} deklarációk és az azokra épülő rutinok. A
korábbi változatokban 
ez a rész hangsúlyosabb volt, mert a futó \code{R} példányok feltérképezése
az  \code{enumwindows} \code{API}
híváson keresztül történt. Ez a rész a modul végén kikommentezve megtalálható.
\begin{description}
\item[\code{sendkeysToHwnd}, \code{sendText}] Az első a \code{Sendkeys}
  függvényt, a második a \code{Sendmessage} hívást  használva küld szöveget
  a megadott ablakba.
\item[\code{GetWndPid}] Ez egy wrapper a
  \code{GetWindowThreadProcessId} API hívás körül. Adott ablak
  \code{pid}jét adja 
  vissza.
\item[\code{getAllWndFromPid}, \code{enumAllbyPid}] Az
  \code{enumWindows} API hívást 
  használva, lekérdezzük az adott processhez tartozó valamennyi
  ablakot.  
\item[\code{GetWndText}] Adott ablak címsorának szövegét adja vissza.

\item[\code{EnumWndLike}, \code{childWndLike}] Adott ablak child
  ablakai között megkeresi 
  az adott mintára illeszkedő címűt. Az összehasonlítást a \code{like}
  operátorral végezzük. Az \code{childwindow}-kat  az \code{enumChildWindows}
  hívással járjuk végig az első találatig.  
\end{description}
Az itt szereplő rutinokat a  \code{RIC} modulból ill. az \code{Rproc}
objektumok inicializálásakor  használjuk. Néhány API függvényt,
közvetlenül meghívunk a  \code{RIC} 
  modulból, pl. \code{sleep}, \code{isWindow}, \code{isWindowVisible},
  \code{showWindow}.

\section{A \code{selectfuns} modul}\label{sec:5.6}

Az \code{Rdev.xlam} bővítmény segítségével \code{File}, \code{ACCESS},
vagy \code{ACCESS + SQL} választó elemeket illeszthetünk be a
munkafüzetekbe. Ezeknél a felhasználói mező jobb felső sarkában
megjelenő kis gombokhoz rendelt makrók az ebben a modulban levő
rutinokat használják. 
Ezeket a makrókat az \code{Rdev.xlam} bővítmény írja a számoló
munkafüzetbe. Pl. ha \code{File} választó mezőt 
szúrunk be, akkor a munkalap makrói között az alábbihoz hasonlóak jelennek meg.
\begin{VBAframe}
Private Sub File2Click()
  selectFile Me.Shapes("Button 2").TopLeftCell, "R,*.R"
End Sub
Private Sub mappa1Click()
  selectmappa Me.Shapes("Button 1").TopLeftCell
End Sub
\end{VBAframe}
A \code{select...} szubrutinok a \code{R.xls} munkafüzetben vannak
definiálva. Jelenleg a következőket használjuk:
\begin{description}
\item[\code{selectMappa(cell As range)}] Könyvtár választó dialógot jelenít
  meg. Ha a felhasználó választ egy könyvtárat, azt a \code{cell} cellába
  írjuk. A kezdő könyvtárat a \code{cell} cellától jobbra lévő cellából
  vesszük.  
\item[\code{selectFile(cell As range, filter As String)}] File
  választó dialógot jelenít meg. A kezdő könyvtár értékét a \code{cell}-től
  jobbra, fent lévő cellából veszi. Ha a felhasználó a \code{Mégse}
  gombot nyomja meg, akkor nem változtat a \code{cell} cella tartalmán. Ha a
  felhasználó választott egy file-t, akkor annak neve, az könyvtár
  elérési útja nélkül a \code{cell} cellába kerül, míg a \code{file}-t
  tartalmazó könyvtár elérési útja a \code{cell} cella fölötti cellába íródik.
\item[\code{selectMDB(cell As range)}] A \code{selectFile} rutint hívja meg
  \code{"Access,*.mdb"} filter értékkel.
\item[\code{selectTBL(cell As range)}] Access
  tábla választó dialógot jelenít meg. Ehhez a \code{Pick} modul
  \code{PickACCESSTable} függvényét használjuk. Az \code{Access}
  adatbázis elérési 
  útját a cell cellától jobbra fent lévő két cella értékéből rakja
  össze.
\item[\code{selectSQL(cell As range, bl1, bl2)}] Ez a szubrutin az
  \code{ACCESS + SQL} választó mezőkben  használatos. A \code{cell}
  cella alatti \code{bl1} 
  számú sort elrejti ha az látható és fordítva, majd az ez alatti \code{bl2}
  számú sort elrejti ha az első \code{bl1} sor látható és fordítva.  
\end{description}


\section{Egyebek}\label{sec:5.7}

\subsection{\code{formadjust} modul}
A \code{selectR userform} beállítására szolgál. Valójában nincs rá
szükség az \code{R.xls} munkafüzetben, viszont így volt a legegyszerűbb a
\code{selectR} űrlap elemeit egymáshoz igazítani. 


\subsection{\code{Registry} modul}
A modul nagy részét valahonnan másoltam, már nem emlékszem honnan. Az
alapvető registry műveleteket teszi lehetővé: olvasás, írás,
törlés. Mi csak olvasásra használjuk. Erre 
épül a modul \code{RcomRegistered} függvénye, amit az \code{R.xls}
munkafüzet betöltésekor a telepítés 
ellenőrzésére használunk, ill. \code{RFromReg} függvény, ami az
\code{Rgui} program elérési útját számolja ki.

\subsection{\code{Worksheetfunctions} modul}
Egyetlen, munkalapokon is használható függvényt definiál. Az
\code{extAddress}-szel már talákoztunk \aref{sec:1.3} részben. Ennek a
függvénynek két argumentuma van, egy tartomány és \code{quote} 
nevű logikai változó. A tartomány teljes nevét (\code{munkafüzet!munkalap+cella cím}) adja vissza.
Ha a \code{quote=IGAZ}, akkor aposztrófok között, különben anélkül.
Ahhoz, hogy mindig a megfelelő értéket kapjuk az \code{R.xls} munkafüzet az
\code{EXCEL} alkalmazás \code{WorkbookBeforeSave} eseménykezelőjét
kiegészíti. Az történik, hogy végigmegyünk a mentendő munkafüzet lapjain és azokat a cellákat, melyek formulája hasonlít a \code{*extaddress*}
mintára (a \code{like} operátort használva) \code{dirty}ként
megjelöljük. Ezután a munkalapot kiértekljük a \code{calculate}
metódussal. Munkafüzet mentése közben a státuszsorban felvillanó
szöveg jelzi, hogy a \code{eztAddress} függvények ellenőrzése történik.

A megvalósítás úgy történik, hogy a R.xls munkafüzetben egy \code{APP}
nevű \code{Class} modult is létrehozunk. Ebben van egy
\code{WithEvent} kulcsszóval definiált \code{application} objektum, amit 
inicializáláskor \code{Excel.Application} értékre állítunk. Az
iniciálizálás az \code{R.xls} munkafüzet 
megnyitásakor történik meg.
Ez a megoldás csak azt az esetet kezeli, ha a munkafüzet neve, az
\code{EXCEL} mentés másként 
funkciója nyomán változik meg. Ha a munkafüzet neve valamely egyéb
módon változott meg, akkor a megnyitás utáni mentés helyreállítja az \code{extAddress}-re épülő hivatkozásokat.

Egy másik lehetőség a \code{APP} modulban definiálni a \code{WorkbookOpen}
eseménykezelőt is, a fentihez 
hasonló módon.

\endinput

%%% Local Variables: 
%%% mode: latex
%%% TeX-master: "Rxls"
%%% End: 
