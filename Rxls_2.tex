%% -*- mode: poly-noweb+R; ess-indent-level: 2; -*-

\chapter{Részletesebben}
\label{chap:2}

Az elképzelt munkamenet a következő. Van egy \code{R} szkriptünk,
aminek a paraméterezését egy \code{EXCEL} munkafüzetből szeretnénk
megoldani.
% ez elé szeretnénk tenni egy munkafüzetet, ami a paramétereket tartalmazza.

\section{Adatok}
\label{sec:2.1}

Első lépésként azonosítani kell a bemenő adatokat és el kell
készíteni a számoló munkafüzet \code{Adatok} munkalapját. Ha nagyobb
mennyiségű fix paramétert is át kell adni, akkor azokat egy külön 
munkalapon is elhelyezhetjük. 
%lehet a szkriptben vagy létre lehet hozni egy olyan munkalapot az
%\code{EXCEL} munkafüzetben ami 
%ezeket a paramétereket tartalmazza. 
Ha ennek, az extra adatokat
tartalmazó, munkalapnak a 
neve \code{R kód}-dal kezdődik, vagy a \code{(hidden)}-nel végződik,
akkor a véglegesítés során az adott 
munkalap rejtetté válik.

Az adatok átvitelére az \code{XLdata} függvényt célszerű
használni. Ennek részletes leírását lásd 
\aref{sec:7.1} szakaszban. Itt megelégszünk annyival, hogy 
\code{XLdata (n1,n2,range="<range cím>")}
függvény hatására a \code{<range cím>} által meghatározott tartomány
\code{n1} oszlopában lévő értékeket változó névként, az \code{n2}
oszlopban mellette lévő értékeket a változó értékeként értelmezi 
az \code{R}. Ha a név mező üres vagy nem szöveges típusú, akkor az
adott sorhoz nem tartozik 
változó definíció. Ha egy név többször is szerepel, akkor az utolsó sor
változó definíciója 
érvényes. Az \code{n1,n2} értékek megadása nem kötelező, ha viszont
hiányzik, akkor a \code{<range cím>} sztringnek egy legalább két
oszlopos tartományt kell kijelölnie és ennek első két oszlopát fogja
használni a program. A \code{<range cím>} teljes oszlopokra is
mutathat, pl \code{\$A:\$C}. Ebben 
az esetben csak a munkalap \code{UsedRange} tartományát használjuk.

Alapértelmezésben, csak a látható sorokat használjuk és egyesített
cellák esetén a cellában 
látható értéket használjuk.

Abban az esetben, ha a változó név nem megengedett karaktereket is
tartalmaz, pl. szóköz, ékezetes karakterek, számmal kezdődő név
stb. akkor az \code{R} oldalon a változó nevét \code{`} 
(back quote) karakterek közé kell tenni. Pl. az \code{`\# R nevek \#`}
nevű változó értéke (ez az 
\code{Adatok} munkalap 5. sora) \code{"Érték"} lesz, ahogy ez
\apageref{fig:1.2}. % 8. 
oldalon látható.

\section{Kód}
\label{sec:2.2} %. Kód
Az \code{R} szkriptet célszerű két részre bontani: függvény definíciós
részre és az aktuális számolást 
végző részre. Ennek csupán teszteléskor van jelentősége. Ha nem az
történik, amit várunk, 
akkor kénytelenek vagyunk a hibát megkeresni a kódban. Az \code{R kód}
munkalapon kijelölhetjük, hogy melyik függvényt működését akarjuk
ellenőrizni. Ennek az az eredménye, hogy az 
adatok és a függvények beolvasása után a \code{debug (<fv. név>)} is
végrehajtásra kerül és így 
a debugolásra kijelölt függvényt lépésenként lehet kiértékelni.

Célszerű a függvényeket és a számolást végző kódot, mondjuk \code{R studio}-ban, létrehozni
és tesztelni. Ezután a kész kódot elmenteni \code{R} szkriptként (\code{.R}
kiterjesztéssel) majd azt importálni a számoló munkafüzetbe \code{R kódlap}ként. Így is elő fordulhatnak váratlan dolgok, de 
ezek nagy részét kiszűrhetjük, ha a kód megírása során a bemenő
adatokat, vagy is a futási 
környezetet már az \code{EXCEL} adja. Ezalatt azt értem, hogy az első
lépésként létrehozott \code{Adatok} 
munkalapot kitöltjük a teszt paraméterekkel, összekötjük az \code{R}
példányunkat az \code{EXCEL}-lel 
(\code{Rxls menü R <-> EXCEL összekötés}) és ráklikkelünk a
\code{Számolás indítása} feliratú gombra. Mivel még nincs semmilyen
végrehajtandó kód a számoló munkafüzetben, ennek hatására a teszt
adatok, az \code{EXCEL}ben definiált változó nevekkel, átkerülnek az
\code{R}-be. Ebben az \code{R} környezetben érdemes dolgozni.


%%% Local Variables: 
%%% mode: latex
%%% TeX-master: "Rxls"
%%% End: 
