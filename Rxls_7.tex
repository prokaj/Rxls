%% -*- mode: poly-noweb+R; ess-indent-level: 2; -*-
\chapter{Az \code{R} oldal részletesebben}\label{chap:7}

\section{Az \code{Rxls} csomag rutinjai}\label{sec:7.1}

\subsection{\code{XLdata}}

Az \code{XLdata} függvény célja, hogy viszonylag könnyen lehessen a
beállításokat, paramétereket az \code{R}-nek átadni. A függvény
argumentumai:
\begin{Rnw}
<<echo=FALSE,results="hide">>=
suppressMessages(require(formatR))
showfundef<-function (x,...){
  name<-deparse(substitute(x))
  body(x)<-NULL;
  y<-capture.output(.Internal(print.function(x,TRUE,...)));
  cat(y[-(length(y)-0:1)],sep="\n");
  invisible(unclass(x))
}
@
<<prompt=FALSE,render=showfundef>>=
XLdata
@
\end{Rnw}
\begin{description}
\item[\code{c1,c2,c2.offset}] Az első két paraméter a neveket
  ill. értékeket tartalmazó oszlopok sorszáma a tartományon belül, a
  harmadik ezek különbsége. A kód a \code{c2.offset} értéket 
  használja, így a \code{c2} értéket csak akkor használjuk, ha
  \code{c2.offset} nincs megadva. 
\item[\code{range}] A tartomány címe, lehetőleg teljes név, azaz
  munkafüzet, munkalap és azon belül a tartomány bal felső és jobb
  alsó cellája. pl. \code{[Munkafüzet1]Munka1!\$A:\$F}. Az \code{extAddress} 
  munkalapfüggvénnyel könnyen megadható.
\item[\code{rmHidden}] Ha \code{TRUE}, akkor a rejtett sorokban lévő
  változókat töröljük az \code{R} oldalon. Alapértelmezésben nem törlünk.
\item[\code{onlyvisible}] Ha \code{TRUE} (ez az alapértelmezés), akkor
  csak a látható sorokat használja, ellenkező esetben a rejtetteket
  is. 
\item[\code{env}] Az környezet (\code{environment}), amiben a változók
  definiálva lesznek. Alapértelmezésben az a függvény hívásakor aktív
  környezet. Ha a \code{background(eval(.FUN))} vagy \code{eval(.FUN)}
  részeként használjuk az \code{XLdata} függvényt, akkor a globális
  \code{.GlobalEnv} környezetben lesznek definiálva a
  változók. Tipikusan ezt a paramétert nem kell megadni. 
\item[\code{XLrange}] A tartomány címének megadása helyett egy \code{EXCEL}
  \code{range}-t reprezentáló \code{COM} objektumot is
  megadhatunk. Tipikusan ezt a paramétert nem kell megadni.
\item[\code{XL}] A használandó EXCEL példány. Tipikusan ezt a
  paramétert nem kell megadni. 
\item[\code{value}] A cella melyik érték tulajdonságát használjuk?
  \code{"value"} vagy \code{"value2"}. Csak a dátum/idő adatok
  kezelésében van különbség.  \code{"value"} esetében az EXCEL dátum/idő adatai
  \code{POSIXt} típusú adatként  jelennek meg az \code{R} oldalon, míg
  \code{"value2"} esetében az \code{1900.01.01.}
  dátum óta eltelt napok számaként. A \code{POSIXt} típus konverziót a
  téli/nyári időszámítás  
  összezavarja, ezért inkább a másikat használtam mindig. \code{"value2"}
  használata esetén az 
  eredmény az \code{XLDate} függvény segítségével \code{Date} típusúvá
  alakítható. Tipikusan ezt a 
  paramétert nem kell megadni. 
\end{description}

\subsection{\code{XLsource}}

Ez a függvény a \code{source} mintájára működik, azaz a kódot olvas be
szövegként, majd azt elemzi (\code{parse}) és kiértékeli, vagyis
végrehajtja. Azonban a \code{source}-től eltérően az \code{XLsource} 
nem file-ból, hanem egy \code{EXCEL} munkafüzet \code{R kódlap}járól olvassa
be a kód szövegét. Az argumentumok:
\begin{Rnw}
<<prompt=FALSE,render=showfundef>>=
XLsource
@
\end{Rnw}
% XLsource
% function (..., wbname, name = {
% dots <- match.call(expand.dots = FALSE)$...
% if (length(dots) && !all(sapply(dots, function(x) is.symbol(x) ||
% is.character(x))))
% stop("... must contain names or character strings")
% sapply(dots, as.character)
% }, env = parent.frame(), wsname = sprintf("R kód(%s)", name),
% wb = {
% on.exit(wb <- NULL, add = TRUE)
% if (missing(wbname))
% XL[["activeworkbook"]]
% else XL[["workbooks", wbname]]
% }, XL = THISXL)
\begin{description}
\item[\code{...}] Az \code{R kódlap}-ok nevei, {\em nem szükséges
  idézőjelek közé tenni a neveket}. Ha a nevek karakter vektor
  formájában állnak rendelkezésre, akkor használjuk a \code{name}
  argumentumot. Ezek a nevek nem a munkalap nevek. Azok 
  \code{R kód(kódlap név)} alakúak, lásd a \code{wsname} argumentumot.
\item[\code{wbname}] A munkafüzet neve. Ha nem adjuk meg, akkor az
  aktív munkafüzetet fogja a rutin használni.
\item[\code{name}] A kódlapok neveit \code{c ("<név1>","<név2>",...)}
  alakban is megadhatjuk. Tipikusan ezt az argumentumot nem kell megadni.
\item[\code{env}] Hasonlóan az \code{XLdata} függvényhez itt is
  megadható, hogy melyik környezetben dolgozunk. Ugyanaz a mechanizmus
  érvényesül itt is. Tipikusan ezt az argumentumot nem kell megadni.
\item[\code{wsname}] Az \code{R} kódot tartalmazó munkalapok nevei. Tipikusan
  ezt az argumentumot nem kell megadni.
\item[\code{wb}] A munkafüzetet reprezentáló \code{COM}
  objektum. Tipikusan ezt az argumentumot nem kell megadni.
\item[\code{XL}] A használni kívánt EXCEL példány. Tipikusan ezt az argumentumot nem kell megadni.
\end{description}

\subsection{\code{XLwritedf}}

Egy \code{data.frame}-t másol \code{EXCEL} munkafüzetbe. Külön figyelni kell
a \code{factor} és dátum típusú 
adatokra. Lehetőség van az oszlop szélesség tartalomhoz igazítására és
név megadására. 
\begin{Rnw}
<<prompt=FALSE,render=showfundef>>=
XLwritedf
@
\end{Rnw}
% XLwritedf
% function (XLrange, df, with.names = TRUE, autoFit = TRUE,
% setname)
\begin{description}
\item[\code{XLrange}] A céltartomány. A \code{df} \code{data.frame}-t
  az \code{XLrange} tartomány bal felső sarkába illesztjük be. 
\item[\code{df}] A másolandó \code{data.frame}.
\item[\code{with.names}] Ha \code{TRUE}, akkor a kitöltött tartomány
  első sorába a \code{data.frame} nevei kerülnek. 
\item[\code{autoFit}] Ha \code{TRUE}, akkor az oszlopok szélességét a
  tartalomhoz igazítjuk. 
\item[\code{setname}] Ha megadjuk és értéke karakterlánc, akkor ez
  lesz a kitöltött tartomány (munkafüzet szintű) neve.
\end{description}

\subsection{\code{XLreaddf.cols}, \code{XLreaddf.rows}}

Mindkét függvény az \code{EXCEL} megadott tartományát alakítja R
\code{data.frame}-é. A különbség 
abban van, hogy míg a \code{XLreaddf.cols} esetén a tartomány oszlopai
felelnek meg az eredmény  
\code{data.frame} oszlopainak, addig az \code{XLreaddf.rows} esetében
a tartomány soraiból lesznek a \code{data.frame} 
oszlopai. 
Az előfeldolgozás (beolvasás, összevont cellák ellenőrzése) után
mindkét rutin az \code{XLlist2df} 
függvényt használja és \code{...} argumentumot tovább adja ennek a
függvénynek. Így a paraméterek listája hosszabb, mint ami az alábbi
fejlécből következne.  
\begin{Rnw}
<<prompt=FALSE,render=showfundef>>=
XLreaddf.cols
XLreaddf.rows
@
\end{Rnw}
% XLreaddf.cols
% function (range, which.cols = seq_len(ncol(x)), XLrange = {
% on.exit(XLrange <- NULL, add = TRUE)
% if (missing(range))
% XL$selection()
% else XL$range(range)
% }, ..., chk.merge = TRUE, value = "value2", XL = THISXL)
% XLreaddf.rows
% function (range, which = seq_len(nrow(x)), XLrange = {
% on.exit(XLrange <- NULL, add = TRUE)
% if (missing(range))
% XL$selection()
% else XL$range(range)
% }, ..., chk.merge = TRUE, value = "value2", XL = THISXL)
\begin{description}
\item[\code{range}] A céltartomány szövegesen megadott címe. Ha nem
  adjuk meg, akkor az aktuális kijelölést használjuk.
\item[\code{which.cols}] Itt lehet megadni, hogy a tartomány mely
  oszlopai szerepeljenek a végeredményben. Alapértelmezése a tartomány
  összes oszlopa. 
\item[\code{XLrange}] A tartomány \code{COM} objektumként. Jellemzően nem kell megadni.
\item[\code{...}] Ezeket a paramétereket tovább adjuk a
  \code{XLlist2df} függvénynek. 
  \begin{Rnw}
%<<echo=FALSE,results="hide">>=
%XLlist2df<-Rxls:::XLlist2df;class (XLlist2df)<-"fundef"
%@

<<prompt=FALSE,render=showfundef>>=
Rxls:::XLlist2df
@
  \end{Rnw}
% Rxls:::XLlist2df
% function (x, header = 1, trim = FALSE, skip = 0, head.sep = "",
% numeric = FALSE, ...)
Az így keletkező további beállítási lehetőségek:
\begin{description}
\item[\code{header}] A tartomány fejlécének nagysága, sorokban.
\item[\code{trim}] Ha igaz, akkor az üres sorokat és oszlopokat
  eldobjuk, ha hamis nem történik semmi.
\item[\code{skip}] Lehet függvény, ekkor a trimmelés után ezt
  alkalmazzuk az adatokra. Lehet szám is. Pozitív érték esetén \code{skip}
  számú sort eldobunk. 
\item[\code{head.sep}] Akkor használjuk, ha a tartomány fejléce több
  soros. Ilyenkor a \code{data.frame} 
neveit az oszlop fejlécében szereplő sztringek összefűzésével kapjuk.
\item[\code{numeric}] Ha igaz, akkor a \code{data.frame} elemeit
  megpróbáljuk numerikus értékre konvertálni akkor is, ha mondjuk az
  \code{EXCEL}-ben szövegesen szerepelnek a számok. A tizedesjel az
  \code{R} tizedes jele, alapértelmezésben ez a pont. 
\item[\code{...}] Ezeket a paramétereket továbbadjuk a \code{data.frame}
  függvénynek. Célszerű lehet a  \code{stringAsFactors = FALSE}
  megadása. Ugyanis 
  szöveges adatból \code{data.frame} létrehozása során \code{factor}
  változó lesz. Ha mégis inkább szöveges adatot szeretnénk a
  \code{data.frame}-ünkben, 
  akkor ezt a beállítást kell használni.
\end{description}
Először trimmelünk ha kell, majd a \code{skip}et alkalmazzuk, ezután
választjuk le a fejlécet 
a \code{header} alapján. A megmaradó adatokra alkalmazzuk a
\code{as.numeric}  függvényt, ha
szükséges, majd a \code{data.frame} függvényt. Végül beállítjuk a
\code{data.frame} neveit, ha vannak.

\item[\code{chk.merge}] Ha a tartományban összevont cellák vannak,
  akkor a képernyőn látható érték az összevont cellák bal felső
  sarkában lévő cella értéke, a többi cella értéke üres. Ez a \code{R} 
  oldalon \code{NA} értéket eredményez. Ha \code{chk.merge} értéke
  \code{TRUE}, akkor az \code{R} oldalra másolt 
  értékek megegyeznek a cellában látható értékkel. Vagyis egy
  összevont tartomány minden cellájához a képernyőn látható értéket
  rendeljük. 
\item[\code{value}] Melyik tulajdonságát használjuk a
  tartománynak. Tipikusan nem kell megadni. Lehet 
  \code{"value"}, \code{"value2"}, \code{"formula"} vagy akár
  \code{"formulalocal"}. 
\item[\code{XL}] Melyik \code{EXCEL} példánnyal dolgozzunk. Tipikusan nem
  kell megadni. Alapértelmezésben a globális \code{THISXL} érték.
\end{description}

\subsection{\code{XLget}}
Egy \code{EXCEL} tartományt olvas be \code{R}-be, pontosabban a tartomány és az
adott munkalap usedrangeének metszetét. 
\begin{Rnw}
<<prompt=FALSE,render=showfundef>>=
XLget
@  
\end{Rnw}
% XLget
% function (range, XLrange = {
% if (missing(range))
% XL$selection()
% else XL$range(range)
% }, value = "value", XL = THISXL)
\begin{description}
\item[\code{range}] A tartomány címe, sztring. Lehet az aktív
  munkafüzetben definiált név is. Ha nem adjuk meg, akkor a kijelölt tartomány.
\item[\code{XLrange}] Az olvasandó tartományt reprezentáló \code{COM}
  objektum. Tipikusan nem kell megadni. 
\item[\code{value}] Melyik tulajdonságát használjuk a
  tartománynak. Tipikusan nem kell megadni. Lehet 
  \code{"value"}, \code{"value2"}, \code{"formula"} vagy akár \code{"formulalocal"}.
\item[\code{XL}] Melyik \code{EXCEL} példánnyal dolgozzunk. Tipikusan nem
  kell megadni. Alapértelmezésben a globális \code{THISXL} érték.
\end{description}

\subsection{\code{ACCESS.get}}
ACCESS adatbázisból olvas be egy táblát.
\begin{Rnw}
<<prompt=FALSE,render=showfundef>>=
ACCESS.get
@
\end{Rnw}
% ACCESS.get
% function (dir, file, table, where = "")
\begin{description}
\item[\code{dir}] Az adatbázis könyvtára.
\item[\code{file}] Az adatbázis fileneve.
\item[\code{table}] A tábla neve.
\item[\code{where}] A lekérdezés where része. Ha nem adjuk meg a teljes táblát lekérdezzük.
\end{description}

\subsection{\code{ODBC.get}}
ODBC adatbázisból olvas be egy táblát.
\begin{Rnw}
<<prompt=FALSE,render=showfundef>>=
ODBC.get
@
\end{Rnw}
% ODBC.get
% function (dsn, sql)
\begin{description}
\item[\code{dsn}] A kapcsolatot leíró sztring.
\item[\code{sql}] Az lekérdezés sql parancsa.
\end{description}

\subsection{\code{getFromDB}}

Választ a két előbbi rutin közül. Ha a \code{dsn} argumentumként nem
definiált változót adunk át, 
akkor az \code{ACCESS.get} egyébként a \code{ODBC.get} függvényt használjuk.

\begin{Rnw}
<<echo=FALSE,results="hide">>=
##if(exists("getFromDB")) rm(getFromDB)
@
<<prompt=FALSE,render=showfundef>>=
getFromDB
@
\end{Rnw}
% getFromDB
% function (dir, file, table, where = "", dsn, sql)
% {
% if (!inherits(try(dsn, silent = TRUE), "try-error")) {
% ODBC.get(dsn = dsn, sql = sql)
% }
% else {
% ACCESS.get(dir = dir, file = file, table = table,
% where = where)
% }
% }
% <environment: namespace:Rxls>



\subsection{\code{logDevice} kezelés}
\subsubsection{\code{logDev...}}
A logDev függvény új \code{logDevice}-t hoz létre a megadott
típussal. \code{A logDev.set} függvény 
az aktuális \code{logDevice}-t állítja át, \code{logDev.off}
kikapcsolja azt, míg a 
\code{logDev.list} kilistázza a létező \code{logDevice}-okat.
\begin{Rnw}
<<prompt=FALSE,render=showfundef>>=
logDev
logDev.set
@
\end{Rnw}
\begin{description}
\item[\code{...}]Az új \code{logDevice} típusa. \code{"null"}, \code{"R"}, \code{"EXCEL"}, \code{"TCL"} lehet.
\item[\code{num}]Az új aktív eszköz sorszáma. 
\end{description}
% logDev
% function (...)

% Átállítja az aktuális logDevice-t.
% logDev.set
% function (num)
% num Az új aktív eszköz sorszáma.

\subsubsection{\code{progress}}

Új \code{progressbar}t ad az aktív \code{logDevice}hoz.
\begin{Rnw}
<<prompt=FALSE,render=showfundef>>=
progress
@
\end{Rnw}
A paramétereket továbbadjuk. A tényleges argumentumok:
\begin{description}
\item[\code{pattern}] Az \code{sprintf} függvény által használt
  formátum sztring. 
\item[\code{...}] A formátum sztringbe helyettesítendő kezdő értékek.
\item[\code{lazyness}] Numerikus érték, a progressbar frissítési
  időköze másodpercben. 
\end{description}

A progress visszatérési értéke egy objektum (list) a következő függvényekkel:
\begin{description}
\item[\code{inc()}] Ha progressbar mintájának egyetlen numerikus
  argumentuma van, akkor azt egyel növeli és frissíti a
  \code{logDevice}t a \code{lazyness} paraméterrel összhangban. 
\item[\code{step(...)}] A progressbar értékét frissíti a
  \code{sprintf(pattern,...)} függvényhívással és frissíti a
  \code{logDevice}t a \code{lazyness} paraméterrel összhangban. 
\item[\code{hide()}] Ideiglenesen leveszi a \code{progressbar}-t a
  \code{logDevice}-ról. Nem frissít! 
\item[\code{show()}] Visszateszi a \code{progressbar}-t a
  \code{logDevice}-ra. Nem frissít! 
\item[\code{reset(pattern,...,lazyness)}] A \code{progressbar} paramétereit
  átállítja. Nem frissít! 
\item[\code{refresh()}] Frissíti a \code{logDevice} szövegét, függetlenül az
előző frissítés óta eltelt időtől. 
\item[\code{rm()}] Véglegesen törli a progressbart a \code{logDevice}-ról. Nem frissít.
\end{description}


\subsubsection{\code{logMessage}}

\begin{Rnw}
<<prompt=FALSE,render=showfundef>>=
logMessage
@
\end{Rnw}
%
%function (..., sleep = 0)
\begin{description}
\item[\code{...}] Az \code{sprintf} függvénynek átadott paraméterek,
  az első elem lehet formátum sztring. 
\item[\code{sleep}] Várakozás a \code{message} megjelenítése után másodpercben.
\end{description}

\subsection{Dátum kezelő függvények}
Az \code{EXCEL}-ből \code{"value2"} tulajdonságon keresztül kapott numerikus
értéket az \code{R} valamelyik dátum 
típusára konvertáló függvények.
\begin{Rnw}
<<prompt=FALSE,render=showfundef>>=
XLDate
XLPOSIXct
@
\end{Rnw}
\begin{description}
\item[\code{date}] Numerikus érték. A kezdő időpont óta eltelt napok
  száma. Nem feltétlenül egész szám. 
\item[\code{d1904}] Kezdő időpont. Az \code{EXCEL} munkafüzetek
  \code{Date1904} nevű 
  tulajdonságának felel meg. 
\end{description}

\subsection{Egyebek}
\subsubsection{\code{XLScreenUpdateTurnOff}}

Kikapcsolja az \code{EXCEL} \code{ScreenUpdate} tulajdonságát és a
\code{Calculation} tulajdonság értékét az  \code{xlCalculationManual}
értékre állítja. A visszatérési érték egy függvény, ami a
visszaállítja az eredeti állapotot. Lehetőség 
van az aktív cella értékének visszaállítására is.
\begin{Rnw}
<<prompt=FALSE,render=showfundef>>=
XLScreenUpdateTurnOff
@
\end{Rnw}
%function (Appl, activecell = FALSE)
\begin{description}
\item[\code{Appl}] Az az \code{EXCEL} application, aminek az képernyő
  frissítését le akarjuk tiltani. 
\item[\code{activecell}] Ha igaz, akkor az aktív cella helyét is
  megjegyezzük és a helyreállítás során a cellát aktiváljuk. Csak
  akkor állítsuk igazra, ha időközben nem töröljük a cella munkalapját
  vagy munkafüzetét. 
\end{description}

\subsubsection{\code{XLusedrange}}

Visszaadja egy tartománynak és a \code{usedrange} tartománynak a
metszetét. Az eredmény egy \code{range} objektum.
\begin{Rnw}
<<prompt=FALSE,render=showfundef>>=
XLusedrange
@
\end{Rnw}
%function (r)
\begin{description}
\item[\code{r}] Az a \code{range} objektum, (tehát nem szöveges cím)
  amit a \code{usedrange} tartománnyal metszeni akarunk.
\end{description}

\subsubsection{\code{pkgzip}}
\code{win-binary} formátumban ment telepített \code{R} csomagokat.
\begin{Rnw}
<<prompt=FALSE,render=showfundef>>=
pkgzip
@
\end{Rnw}
% function (pkgs = {
% preselect <- grep("(com(|proxy)|Rxls)$", rownames(ipkgs),
% value = TRUE)
% if (ask)
% select.list(rownames(ipkgs), preselect, multiple = TRUE,
% title = "Zippelendő csomagok:", graphics = T)
% else preselect
% }, zip.mappa = choose.dir(Sys.getenv("HOME"), "Zip állományok helye:"),
% ask = interactive(), PACKAGES = FALSE)
\begin{description}
\item[\code{pkgs}] Az \code{R} csomagok listája. Ha nem adjuk meg és
  az \code{ask} 
  paraméter igaz, akkor interaktív módon választhatunk a telepített
  csomag listájából. Ha hiányzik és az \code{ask} paraméter 
  hamis, akkor értéke \code{c("com","comproxy","Rxls")}.
\item[\code{zip.mappa}] A zippelt csomagok helye. Ha nem adjuk meg, akkor egy könyvtár-választó dialóg segítségével jelölhetjük ki.
\item[\code{ask}] Megkérdezzük-e a felhasználót?
\end{description}

\subsubsection{\code{installRxls}}

Az \code{Rxls} csomagban lévő \code{EXCEL} fileokat átmásolja a
\code{<appdata>/Rxls} könyvtárba és a verzió információkat a \code{versions}
nevű fileba írja. 
\begin{Rnw}
<<prompt=FALSE,render=showfundef>>=
installRxls
@
\end{Rnw}
%function ()

\subsubsection{\code{R\_RegEntries}}
Az \code{R}-rel kapcsolatos regisztry bejegyzésket listázza ki. Ezek
egy része a \code{Software/R-core}
kulcs alatt található. Egy másik része a \code{com} csomag által bejegyzett
\code{CLSID} és \code{Typelib} adat. 
\begin{Rnw}
<<prompt=FALSE,render=showfundef>>=
R_RegEntries
@
\end{Rnw}

\begin{Rnw}
<<echo=FALSE,results="hide">>=
fns<-ls()
rm(list=fns[sapply(sapply(fns,get,envir=environment()),inherits,"fundef")])
rm (fns)
@
\end{Rnw}
\section{\code{VBA} kód konvertálása \code{R} kóddá}\label{sec:7.2}

Szükség lehet arra, hogy \code{R}-ben írjunk meg olyan programrészeket, amik
\code{EXCEL VBA}-ban már 
megvannak, vagy ott könnyen előállíthatóak, mondjuk a makrórögzítés
funkció segítségével. 
A konverzió nem túl komplikált.
A konstansokat kigyűjthetjük a Visual Basic szerkesztő Object browser-éből.
Ahol \code{EXCEL} objektumot használunk (\code{Selection}, \code{ActiveCell},
\code{ActiveSheet}, stb) ott valójában az \code{Application} objektum
tulajdonságaira hívatkozunk, tehát a teljes leírás az \code{VBA}-ban 
\code{Application.Selection} lenne és hasonlóan a többi esetben is.
Visual Basic-ben a \code{.} (pont) operátor egy objektum tulajdonságát
vagy metódusát kérdezi le. Erre az \code{R} oldalon a \code{"[["},
ill. \code{\$} operátorok használhatóak. Az elsővel tulajdonságot, a 
másodikkal metódust vagy tulajdonságot érhetünk el. 
Azaz az \code{EXCEL} \code{workbooks collection}-ját kétféleképpen
is elérhetjük. 
\begin{Rnw}
<<>>=
THISXL[["workbooks"]]
THISXL$workbooks()
@
%$
\end{Rnw}
Ha az első munkafüzetet akarjuk elérni, akkor a Workbooks objektum
item tulajdonságát kellene használnunk. Ez a \code{default}
tulajdonság, tehát a következők valamennyien használhatóak 
\begin{Rnw}
<<eval=FALSE,prompt=FALSE>>=
THISXL[["workbooks", 1]]
THISXL[["workbooks"]][["item", 1]]
THISXL$workbooks(1)
THISXL$workbooks()$item(1)
THISXL[["workbooks"]]$item(1)
THISXL$workbooks()[["item", 1]]
@
%$
\end{Rnw}
Ha metódust akarunk meghívni, akkor csak a \code{\$} szintaxis
használható 
\begin{Rnw}
<<>>=
wb <- THISXL$workbooks()$add() 
@
\end{Rnw}
Ha írható tulajdonságot akarunk beállítani, akkor csak a \code{[[}
szintaxis használható értékadással kombinálva. A következők mind az
\code{ActiveCell} tartomány értékét állítják 1-re. 
\begin{Rnw}
<<>>=
withObj(THISXL,.[["activecell"]]<-1)
withObj(THISXL[["activecell"]],.[["value"]]<-1)
THISXL[["activecell"]] <- 1 
THISXL[["activecell"]][["value"]] <- 1
ac <- THISXL$activecell() 
ac[["value"]] <- 1 
ac <- NULL
@
%$
\end{Rnw}
A segédváltozós megoldásban az \code{ac<-1} nem működne, egyszerűen 1 értéket
adnánk az \code{ac} nevű változónak.

Miután létrehoztunk egy hivatkozást valamely \code{COM} objektumra, használat
után mindig töröljük azt. Ha egy függvény belsejében tesszük ezt akkor
a hivatkozás törlése automatikusan 
megtörténik a függvényből való kilépés után (pontosabban az azt követő
 első garbage collection alkalmával). 
Globális szintű változók használata esetén kézzel kell a változó
értékét \code{NULL}-ra változtatni, vagy 
az \code{rm} függvénnyel törölni a változót magát.
\begin{Rnw}
<<>>=
wb$close(FALSE)
wb <- NULL
gc()
@  
%$
\end{Rnw}
Előfordulhat, hogy nem elég törölni a hivatkozást az adott objektumra,
pl. mert egy új \code{ACCESS}, \code{EXCEL} példányt vagy új
munkafüzetet hoztunk létre, amire a továbbiakban nincs 
szükség. Ha ez egy függvény belsejében történt, akkor jól használható
az \code{R} \code{on.exit} lehetősége. Ezzel olyan kódot tudunk megadni, amit
a függvény kiértékelése után hajt végre az 
\code{R}. A következő kódrészlet a függvény elején létrehoz egy új
munkafüzetet majd kilépéskor 
mentés nélkül bezárja.
\begin{Rnw}
<<eval=FALSE,prompt=FALSE>>=
f <- function() {
  wb <- THISXL$workbooks()$add()
  on.exit(wb$close(FALSE), add = TRUE)
  ## ...
}
@
%$  
\end{Rnw}

\subsection{Egy meglepő jelenség}

Időnként nem pontosan az történik, amit várunk. Tegyük fel, pl. hogy
az \code{activeCell} cellát szeretnénk az \code{=MA()} formulával
ellátni. Kézenfekvő megoldásnak tűnik a következő sor
\begin{Rnw}
<<echo=FALSE,results="hide">>=
ws<-THISXL[["activeworkbook"]][["sheets"]]$add()
ws$range("A1")$activate() ## $
rm(ws)
gc ()
@
<<>>=
THISXL[["activecell"]][["formula"]] <- "=MA()"
@  
\end{Rnw}
Azonban ennek hatására a cella értéke ugyan az aktuális dátum lesz, de
a formula nélkül. 
\begin{Rnw}
<<>>=
THISXL[["activecell"]][["formula"]]
XLDate(THISXL[["activecell"]][["value2"]])
@
\end{Rnw}
Ha azonban beiktatunk egy közbülső lépést, akkor a várt eredményt kapjuk.
\begin{Rnw}
<<>>=
withObj(THISXL[["activecell"]], .[["formula"]] <- "=MA()")
THISXL[["activecell"]][["formula"]]
XLDate(THISXL[["activecell"]][["value2"]])
@  
\end{Rnw}
A különös viselkedés megértéséhez kiírjuk a végrehajtott lépések részleteit.
\begin{Rnw}
<<>>=
`[[.COMObject` <- function(handle, property, ...) {
  print(match.call())
  cat("\thandle =", deparse(handle), "\n")
  x <- com::`[[.COMObject`(handle, property, ...)
  cat("\treturn value =", deparse(x), "\n\n")
  x
}
`[[<-.COMObject` <- function(handle, property, ...) {
  print(match.call())
  cat("\thandle =", deparse(handle), "\n")
  x <- com::`[[<-.COMObject`(handle, property, ...)
  cat("\treturn value =", deparse(x), "\n\n")
  x
}
THISXL[["activecell"]][["formula"]] <- "=MA()"  
@
\end{Rnw}
Azaz az történt, hogy először a \code{THISXL} objektum
\code{activeCell} tulajdonságát kérdeztük le, 
majd az így kapott \code{COM} objektum formula tulajdonságát
beállítottuk. Ezután a változásokat ``átvezettük'' a \code{THISXL} objektumra
is. Pontosabban az utolsó lépésben \code{THISXL} \code{activecell}-jét
a második lépésben kapott eredményre állítjuk, ami ugyanez az aktív
cella. Csakhogy ez 
azt eredményezi, hogy mindkét oldalon a \code{COM} objektum
\code{default} tulajdonságát használjuk, 
vagyis valójában az \code{activecell} \code{value} tulajdonságát
állítjuk saját magára. Visual Basicben 
kifejezve az alábbi kódnak megfelelő lépések történnek:
\begin{VBAframe}
set tmp = Application.activeCell
tmp.formula = "=MA()"
Application.activeCell = tmp
\end{VBAframe}
Ahol az utolsó sor azzal ekvivalens, hogy
\begin{VBAframe}
Application.activeCell.value = tmp.value
\end{VBAframe}
Ez az utolsó lépés az, ami a formulát eltünteti és helyettesíti az
aktuális értékkel. Amikor a 
második változatot használtuk, ahol egy segédváltozóban tároltuk az
\code{ActiveCell} objektumot, erre a lépésre nem került sor. 

Összefoglalva, ha valamilyen tulajdonságot állítunk be a \code{COM} interfész
segítségével, célszerű a segédváltozókat használni, hogy a fenti
példában látott mellékhatást elkerüljük. 

Az \code{R language definition} leírás 3.4.4 \code{Subset assignment}
című részében részletesen 
olvashatunk az értékadó függvényekről.


\section{Dokumentáció, \code{vignette}-ák}\label{sec:7.3}
Valamennyi általam készített R csomag esetén a kommentezett kód
böngészhető pdf formátumban. Pl. az \code{Rxls} csomag esetén a következő
utasítást kell kiadni R-ben. 
\begin{Rnw}
<<eval=FALSE>>=
vignette("Rxls")
@  
\end{Rnw}
A megjelenítéshez szükséges, hogy a file társítások között
regisztrálva legyen pdf nézegető program, pl. Acrobat Reader. 

\code{RStudio} használata esetén, mind az \code{R} konzolban, mind a
szövegszerkesztő ablakban elérhető a kód ``kiegészítése'' funkció a
\code{TAB} billentyű lenyomásával. További előnye az \code{RStudio} 
használatának, hogy a változó nevek kiegészítése, mellett azok rövid
leírása is megjelenik \code{tooltip}-szerűen.

Ha egy adott \code{R} függvény nevére nem emlékszünk pontosan, akkor
jól használható a \code{apropos} függvény, ami a megadott mintára illeszkedő
függvényneveket sorolja fel. 

%%rm(print.function,pos=grep("functions:knitall",search())[1])
\begin{Rnw}
<<echo=FALSE,result='hide'>>=
if(exists("THISXL",inherits=FALSE)) rm (THISXL)
@  
\end{Rnw}
%%%%%%%%%%%%%%%%%%%%%%%%%%%%%%%%%%%%%%%%%%%%%%%%%%%%%%%%%%%%%%%%%%%%%%%%%%%%%%
%%%%%%%%%%%%%%%%%%%%%%%%%%%%%%%%%%%%%%%%%%%%%%%%%%%%%%%%%%%%%%%%%%%%%%%%%%%%%%
%%%%%%%%%%%%%%%%%%%%%%%%%%%%%%%%%%%%%%%%%%%%%%%%%%%%%%%%%%%%%%%%%%%%%%%%%%%%%%



%%% Local Variables: 
%%% mode: latex
%%% TeX-master: "Rxls"
%%% End: 
