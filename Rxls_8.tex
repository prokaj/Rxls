%% -*- mode: poly-noweb+R; ess-indent-level: 2; -*-

\chapter{Hibaelhárítás}\label{chap:8}

\section{\code{COM} problémák, \code{DebugView}}\label{sec:8.1} 

 
Előfordulhat, hogy egy látszólag hibátlan COM interfészt használó
hívás esetén nem az történik amit várunk, a legtöbb ilyen esetben a
visszatérési érték \code{NULL}. A \code{com} csomag nem ír ki hiba
üzeneteket az R konzolra. Üzeneteit a 
DebugView programmal lehet megnézni. Ezt a programot a Microsoft
honlapjáról lehet letölteni:
\begin{center}
  \url{http://technet.microsoft.com/en-us/sysinternals/bb896647.aspx}.
\end{center}
Futtatásához rendszergazda jogosultság kell.

\section{\code{R} kód debugolása}\label{sec:8.2}
Ha az R kódunk hibát (\code{error}), vagy akár csak figyelmeztetést
(warning)  generál, akkor célszerű
megkeresni az okát. A legelső eszköz amit itt használhatunk a
\code{traceback} függvény, ami a 
hiba helyének behatárolásához ad segítséget. Ha a hibát generáló
függvény megvan, akkor 
használhatjuk a \code{debug} függvényt. Hatására az argumentumában
megadott függvény végrehajtását lépésenként tudjuk nyomon követni. 

Ha egy megadott függvény adott pontjától 
szeretnénk ugyanezt tenni, akkor a kódba ideiglenes beszúrhatjuk a
\code{browser()} függvényhívást. Részletesebben lásd az \code{R
  help}-jében,  a \code{Writing R Extensions} leírás 4., \code{Debugging}
fejezetében. Ugyancsak olvashatunk az R debugolásáról az \code{R language
definition} leírás 9. 
fejezetében.

Emellett célszerű lehet kipróbálni a \code{debug} csomagot. Ez hasonló
felületet nyújt, mint az 
\code{EXCEL} Visual Basic szerkesztője, azaz a debugolt kód külön ablakban
jelenik meg, és színek 
jelzik, hogy hol tart a kiértékelés.

Az \code{RStudio} környezet a \code{0.98}-as verziótól kezdve szintén
tartalmaz hibakeresési támogatást. 
%Ez a változat jelenleg
%(2013. december) preview releaseként érhető el az alábbi címről:
%\begin{center}
%  \url{http://www.rstudio.com/ide/download/preview}
%\end{center}


%%% Local Variables: 
%%% mode: latex
%%% TeX-master: "Rxls"
%%% End: 
